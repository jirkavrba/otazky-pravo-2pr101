\documentclass[a4paper]{article}

\usepackage[utf8]{inputenc}
\usepackage[czech]{babel}
\usepackage[T1]{fontenc}

\setcounter{tocdepth}{4}
\setcounter{secnumdepth}{4}

\usepackage{graphicx}
\graphicspath{ {./images/} }

\usepackage[colorlinks = true, linkcolor = black, urlcolor = blue]{hyperref}

\usepackage{titling}
\newcommand{\subtitle}[1]{%
  \posttitle{%
    \par\end{center}
    \begin{center}\large#1\end{center}
    \vskip0.5em}%
}

\title{Zpracované otázky k ústní zkoušce z předmětu 2PR101 Právo}
\subtitle{Fakulta informatiky a statistiky, VŠE v Praze}
\author{Jiří Vrba, 2021}
\date{
    \LaTeX{} zdrojový kód je dostupný na \\
    \url{https://github.com/jirkavrba/otazky-pravo-2pr101} \\
    Jakékoliv opravy nebo doplnění jsou velmi vítány (PR)
}

\begin{document}

\maketitle
\clearpage

\tableofcontents
\clearpage

\section{Právo \\ \small{Pojem, objektivní a subjektivní smysl, systém práva}}
\subsection{Pojem}
Normativní regulace společnosti lidí (systém skládající se z příklazů, zákazů a povolení).
\subsection{Objektivní a subjektivní smysl}
\begin{itemize}
    \item \textbf{Objektivní} - soubor pravidel chování, vytvořený státem v určité době a ve speciální formě. Tato pravidlo jsou obecně závazná a státem vynutitelná.
    \item \textbf{Subjektivní} - souhrn oprávnění účastníka právního vztahu, možnost chování zaručená objektivním právem.
\end{itemize}

\subsection{Systém práva}
\subsubsection{Právo veřejné a soukromé}
\begin{itemize}
    \item \textbf{Veřejné} - normy upravující vztahy mezi státem a občany \\
        Znaky: asymetrie, vrchnostenská regulace. \\
        Patří sem právo:
        \begin{itemize}
            \item \textbf{ústavní}
            \item \textbf{správní} - veřejná správa
            \item \textbf{finanční} - státní rozpočet, emise peněz
            \item \textbf{trestní}
        \end{itemize}
        
    \item \textbf{Soukromé} - vztahy subjektů jsou si rovny a vzájemně nezávislé \\
        Patří sem právo:
        \begin{itemize}
            \item \textbf{občanské} - občanský zákoník, majetkové vztahy FO a PO
            \item \textbf{obchodní} - vztahy mezi podnikateli
            \item \textbf{rodinné}
        \end{itemize}
        
    \item \textbf{Právo se smíšenou povahou}
        \begin{itemize}
            \item \textbf{Pracovní právo} - veřejnoprávně upravená výše minimální mzdy a 
            soukromoprávně upravené smlouvy
        \end{itemize}
\end{itemize}

\subsubsection{Právo hmotné a procesní}
\begin{itemize}
    \item \textbf{Hmotné} - soubor norem, které obsahují oprávnění a povinnosti účastníků (subjektů) právních vztahů
    \item \textbf{Procesní} - soubor norem, které stanoví postup, jak se domoci svých hmotných (subjektivních) práv,
    jestliže je hmotné právo porušeno. Slouží k vynucení chování v právních vztazích.
\end{itemize}

\newpage

\subsubsection{Právo mezinárodní a vnitrostátní}
\begin{itemize}
    \item \textbf{Mezinárodní} - upravuje vztahy mezi státy a mezi státy a mezinárodními organizacemi či mezinárodními
    organizacemi navzájem
    \item \textbf{Vnitrostátní} - upravuje vztahy mezi subjekty na území daného státu
\end{itemize}

\subsubsection{Evropské právo}
\par
Lze ho označit jako hybridní - mezinárodní právo veřejné/vnitrostátní právo
\begin{itemize}
    \item \textbf{Komunitární právo (primární)} - zakládací smlouvy, smlouvy o přistoupení
    \begin{itemize}
        \item Pařížská smlouva zakládající ES uhlí a oceli
        \item Římské smlouvy zakládající Euratom, EHS
        \item Jednotný evropský akt \texttt{1986}
        \item Maastrichtská smlouva zakládající EU \texttt{1992}
        \item Amsterdamská smlouva - posílila pravomoci EP \texttt{1997}
        \item Smlouva z Nice - reagovala na plánované rozšíření EU \texttt{2000}
        \item Lisabonská smlouva - přiznala EU vlastní právní subjektivitu
    \end{itemize}
    
    Tvůrci Evropského práva jsou členské státy, patří sem i revize zřizovacích smluv, Listina základních práv EU, protokoly 
    
    \item \textbf{Unijní právo (sekundární)} - účinkem je vnitrostátní, původem je mezinárodní
    \begin{itemize}
        \item \textbf{Nařízení} - všeobecně závazné, nepřevádí se do národních řádů
        \item \textbf{Směrnice} - závazné, jen co do výsledku, který má být dosažen. 
        Převádějí se do národních řádů, při nesplnění hrozí státu zahájení řízení o porušení smlouvy ze strany EU.
        \item \textbf{Rozhodnutí} - je závazné pro toho, komu je určeno (např. konkrétní zemi EU nebo konkrétní společnosti)
        \item \textbf{Doporučení a stanoviska} - nezávazné, patří sem rozsudky Evropského soudního dvora
    \end{itemize}
\end{itemize}

\newpage

\section{Prameny práva obecně a v ČR, \\ legislativní proces}

\subsection{Typy právní kultury}
\begin{itemize}
    \item \textbf{Kontinentálně evropský typ}
    \begin{itemize}
        \item hlavní pramen práva je zákon
        \item odvozen z Římského práva
        \item ČR, země Evropy
    \end{itemize}
    
    \item \textbf{Anglosaský typ}
    \begin{itemize}
        \item převažují precedenty, právní obyčeje, ...
        \item vznikl na základě common law
        \item USA, Spojené království, Kanada
    \end{itemize}
    
    \item \textbf{Islámské právní kultury apod.} 
    \begin{itemize}
        \item náboženské právo
        \item pramenem je např. Korán
    \end{itemize}
\end{itemize}

\vspace{15pt}
\textbf{Jurisprudence} = právní věda

\subsection{Obecně o soustavě pramenů práva}
Prameny práva jsou formy, ve kterých je právo obsaženo.

\subsubsection{Precedenty (precedens)}
\begin{itemize}
    \item individuální právní akt (rozsudek soudu, správního řízení), kterým se řídí případy právními normami, 
    dosud neupravené
    \item rozhodnutí sloužící jako model pro další rozhodování
    \item precendens není používán v ČR
\end{itemize}

\subsubsection{Právní obyčeje}
\begin{itemize}
    \item jde o ,,právo nepsané''
    \item forma živelné formy práva (obyčej původně nemá právotvorný záměr)
    \item vzniklé spontálně na základě dlouhodobé tradice a obecné akceptace veřejností a státem 
    $ \rightarrow $ vzniklo působením času
    
\newpage

    \item aby se obyčej stal pramenem práva, musí být splněno:
    \begin{itemize}
        \item dlouhotrvající masové uskutečňování určitého chování
        \item konkrétně stanovený obsah
        \item uplatňování státního donucení při porušení pravidla
    \end{itemize}
    \item v anglo-americkém systému se právní obyčeje označují jako \textit{common law}
\end{itemize}

\subsubsection{Normativní smlouva}
\begin{itemize}
    \item smlouva jako taková nemá normativní význam, neboť řeší individuální vztahy, ale normativní smlouva pramenem práva je
    \item jsou hlavním pramenem mezinárodního práva
    \item \textbf{Mezinárodní smlouvy}:
    \begin{itemize}
        \item sjednávání a ratifikace mezinárodních smluv jsou v pravomoci prezidenta republiky a u některých smluv se 
        vyžaduje i souhlas Parlamentu ČR
        \item vyhlášeny ve Sbírce mezinárodních smluv ministerstvem zahraničí
        \item přednost před vnitrostátním právem podle \href{https://www.zakonyprolidi.cz/cs/1993-1#cl10}{článku 10 Ústavy ČR} \\
        $ \longrightarrow $ mezinárodní smlouva nemá vyšší právní sílu než zákony, ale má aplikační přednost
    \end{itemize}
    \item \textbf{Kolektivní smlouvy} - upravují pracovní a sociální podmínky zaměstnanců
\end{itemize}

\subsubsection{Normativní právní akt}
Výsledky cílevědomé normotvorné činnosti státních orgánů obsahující právní normy (pravidla chování).
Vznikají legislativním procesem (proces tvorby práva).

\begin{itemize}
    \item prvotní návrh - legislativní záměr
    \begin{itemize}
        \item může podat jen ten, kdo má zákonodárnou iniciativu = vláda, poslanec či skupina poslanců, Senát, vláda, kraj,
        ale prezident ne!
        \item Zpráva, pojednávající o tom, proč by nová norma měla být schválena, jaké instituce zřídit a kolik to bude stát
        \item 3 čtení: Poslanecká sněmovna $ \rightarrow $ Senát $ \rightarrow $ prezident
    \end{itemize}
    \item publikace je provedena registrací ve Sbírce zákonů
\end{itemize}

\subsubsection{Dělení normativních právních aktů}
\begin{itemize}
    \item \textbf{Primární}
    \begin{itemize}
        \item \textbf{zákony} - normativní akt nejvyššího zastupitelského orgánu státu
        \item \textbf{kodexy (zákoníky)} - jediný právní předpis, který komplexně reguluje určité odvětví 
        (např. \href{https://www.zakonyprolidi.cz/cs/2006-262}{zákoník práce})
        \item \textbf{ústavní zákony} - jiná forma
        \begin{itemize}
            \item \textbf{rigidní} - ke schválení je nutná kvalifikovaná většina
            \item \textbf{flexibilní} - schvalují se stejně jako ostatní zákony
        \end{itemize}
        
\newpage
        
        \item \textbf{zákonná opatření Senátu Parlamentu ČR}
        \begin{itemize}
            \item slouží v případech, kdy je rozpuštěna Poslanecká sněmovna
            \item Senát má pravomoc přijímat tato opatření ve věcech, které nesnesou odklad (místo zákonů přijímá opatření)
        \end{itemize}
    \end{itemize}
    \item \textbf{Sekundární} - nelze jimi měnit primární normativní akty, ale lze je rozvíjet
    \begin{itemize}
        \item nařízení vlády
        \item vyhlášky ministerstev
        \item vyhlášky a nařízení orgánů územní samosprávy
    \end{itemize}
\end{itemize}

\subsubsection{Rozhodnutí Ústavního soudu}
Ústavní soud rozhoduje \textit{nálezem} nebo \textit{usnesením}. 
Všechny nálezy jsou publikovány ve Sbírce nálezů Ústavního soudu a mají shodnou právní sílu jako zákon a jsou považovány za pramen práva.
 Příkladem může být Rozhodnutí o zrušení zákonů pro rozpor s ústavním pořádkem.
 
\subsection{Obecné právní zásady}
Představují právní principy, která jsou součástí právního řádu, které mají vysoký stupeň obecnosti.
Např. ,,neznalost zákona neomlouvá'' nebo ,,smlouvy se mají dodržovat''.

\subsection{Prameny práva v ČR}
\begin{itemize}
    \item Právní řád ČR je součástí kontinentálního typu právní kultury
    \item Je založen na psaném právu. Zahrnuje:
    \begin{itemize}
        \item právní předpisy
        \item ratifikované a vyhlášené mezinárodní smlouvy
        \item nálezy Ústavního soudu
    \end{itemize}
\end{itemize}

\subsection{Legislativní proces}
Cílený proces tvorby právních norem
\begin{itemize}
    \item projevuje se především tvorbou právních předpisů
    \item zakotvuje ho zejména Ústava, zákon o jednacím řádu Poslanecké sněmovny, zákon o jednacím řádu Senátu
    \item je to ucelený proces, který má vlastní pravidla a ustálený průběh
    \item začíná analýzou společenských vztahů
    \item \textbf{zákonodárná iniciativa} = legální možnost podat návrh zákona (může jí předložit poslanec, skupina poslanců, Senát, vláda nebo zastupitelstvo vyššího územního samosprávného celku)
    \item návrh zákona se předkládá předsedovi Poslanecké sněmovny
    \item proces přípravy zákona včetně meziresortního připomínkového řízení je upraven v \textbf{legislativních pravidlech vlády}
\end{itemize}

\section{Prameny práva EU, orgány EU}

\subsection{Komunitární (primární) právo}
Primární právo je tvořeno zakládacími smlouvami a smlouvami o přistoupení jednotlivých členských států EU:

\subsubsection{1951 - Pařížská smlouva}
Smlouva o založení Evropského společenství uhlí a oceli = \textbf{ESUO}

\subsubsection{1957 - Římské smlouvy}
\begin{itemize}
    \item Smlouva zakládající Evropské hospodářské společenství = \textbf{EHS}
    \item Smlouva zakládající Evropské společenství pro atomovou energii = \textbf{EURATOM}
\end{itemize}

\subsubsection{1967 - Bruselská smlouva}
Sloučení orgánů ESUO, EHS, a EURATOM $ \rightarow $ vznik \textbf{Evropského společenství (ES)}{}

\subsubsection{1992 - Maastrichtská smlouva}
Vytváří pojem Evropská unie jako jednotné označení pro Evropská společenství.
Zavádí pilířový systém:
\begin{itemize}
    \item první pilíř - Evropská společenství, komunitární otázky
    \item druhý pilíř - Společná zahraniční a bezpečnostní politika
    \item třetí pilíř - Policejní a justiční spolupráce
\end{itemize}

\subsubsection{Revize zřizovacích smluv}
\begin{itemize}
    \item Amsterodamská smlouva
    \item Smlouva z Nice
    \item \textbf{Lisabonská smlouva}
    \begin{itemize}
        \item Lisabonská smlouva zařadila do primárního práva také listinu základních práv a svobod
        \item \textbf{Ruší pilířový systém}
    \end{itemize}
\end{itemize}

\newpage

\subsection{Unijní (sekundární) právo}
\begin{itemize}
    \item \textbf{Nařízení} - všeobecně závazné, nepřevádí se do národních řádů
    \item \textbf{Směrnice} - závazné, jen co do výsledku, který má být dosažen. 
    Převádějí se do národních řádů, při nesplnění hrozí státu zahájení řízení o porušení 
    smlouvy ze strany EU.
    \item \textbf{Rozhodnutí} - je závazné pro toho, komu je určeno (např. konkrétní zemi 
    EU nebo konkrétní společnosti)
    \item \textbf{Doporučení a stanoviska} - nezávazné, patří sem rozsudky Evropského soudního dvora
\end{itemize}

\subsection{Orgány EU}
\subsubsection{Evropská rada}
\begin{itemize}
    \item předsedové vlád nebo hlavy členských států EU + předseda komise
    \item summit = setkání na nejvyšší úrovni
    \item vymezuje obecný politický směr a priority Evropské unie
    \item sídlo v Bruselu (Belgie)
    \item zasedá 4x ročně
    \item má svého předsedu (momentálně Charles Michel)
\end{itemize}

\subsubsection{Rada EU (Rada ministrů)}
\begin{itemize}
    \item představuje zájmy členských států
    \item členové jsou ministři jednotlivých zemí - sjíždí se na základě projednávané problematiky 
    \item systém váženého hlasování - každý stát má určitý počet hlasů (podle rozlohy, ekonomické síly, ...)
    \item hlavní rozhodující orgán společně s EP, schvaluje všechny legislativní akty + rozpočet
    \item sídlo také v Bruselu
\end{itemize}

\subsubsection{Evropská komise}
\begin{itemize}
    \item výkonný orgán EU, představuje zájmy EU 
    \item není závislá na členských státech
    \item \textbf{legislativní iniciativa} - navrhuje nařízení, směrnice a rozhodnutí
    \item svou činností se odpovídá EP, ten jí může vyslovit nedůvěru
    \item jedná neveřejně, aby mohla vystupovat jednotně, prezentuje se až konečný výstup
    \item dohlíží na dodržování komunitárního práva, pokud ho nějaký členský stát poruší, 
    může proti němu zahájit řízení o porušení smlouvy
    \item sídlí v Bruselu, jeden komisionář za každý členský stát EU
\end{itemize}

\subsubsection{Evropský parlament}
\begin{itemize}
    \item sdružování podle politických stran, jediný přímo volený orgán EU
    \item poslanci nereprezentují stát, ale své voliče - křesťany, demokraty...
    \item má 705 poslanců volených na 5 let, poměrným systémem, ČR má 21 poslanců
    \item zasedá ve Štrasburku (Francie), v Bruselu a Lucemburku
    \item \textbf{EP nemá legislativní iniciativu}, pouze schvaluje
\end{itemize}

\subsubsection{Evropský soudní dvůr}
\begin{itemize}
    \item rozhoduje zejména v řízení o porušení smlouvy, vymáhání práva EU
    \item sídlo v Lucemburku
    \item je orgánem soudní moci EU
    \item řeší spory mezi jednotlivými orgány a členskými státy, i státy navzájem
\end{itemize}

\subsubsection{Evropský účetní dvůr}
\begin{itemize}
    \item nejvyšší orgán finanční kontroly EU
    \item sídlo v Lucemburku
\end{itemize}

\subsubsection{Evropská centrální banka}
\begin{itemize}
    \item řídí měnovou politiku EU
    \item sídlo ve Frankfurtu
\end{itemize}

\newpage

\section{Ústavní pořádek ČR, dělba moci \\ \small{Ústava, Listina základních práv a svobod}}

Právní řád ČR je uspořádaný systém pramenů práva platných v ČR nebo v rámci mezinárodního společenství, jehož je ČR členem.
Je součástí \textbf{kontinentálního typu právní kultury} a je založen na psaném právu.

\subsection{Ústava}
\begin{itemize}
    \item základní zákon státu a nejvyšší právní norma jeho právního řádu
    \item právní výraz existence státu
    \item vytváří pravidla výkonu státní moci
    \item zaručuje občanům základní lidská práva
    \item zákony a podzákonné právní předpisy musí být v souladu s Ústavou a ústavními zákony, které dohromady tvoří ústavní řád
    \item na dodržování ústavního řádu dohlíží a případné rozpory prostých zákonů s ním řeší v právních státech ústavní soudy
    \item Ústava ČR - psaná, rigidní = ke změně je potřeba souhlas
    \item základní zákon České republiky - přijat jako ústavní zákon Českou národní radou 16. 12. 1992
    \item obsahuje preambuli a 113 článků rozdělených do 8 hlav
    \begin{enumerate}
        \item základní ustanovení
        \item moc zákonodárná
        \item moc výkonná
        \item moc soudní
        \item NKÚ = Nejvyšší kontrolní úřad
        \item ČNB = Česká národní banka
        \item územní samospráva
        \item přechodná a závěrečná ustanovení
    \end{enumerate}
\end{itemize}

\subsection{Listina základních práv a svobod}
\begin{itemize}
    \item je součástí ústavního pořádku ČR - součást Ústavy ČR přijaté roku 1992
    \item vychází z Všeobecné deklarace lidských práv
    \item práva v této listině jsou nezadatelná, nezcizitelná, nepromlčitelná a nezrušitelná
\end{itemize}

\subsubsection{Právo na život}
\begin{itemize}
    \item po narozen nesmí být života nikdo zbaven, v ČR zrušen i trest smrti
\end{itemize}

\subsubsection{Nedotknutelnost osoby a jejího soukromí}
\begin{itemize}
    \item nikdo nesmí být bezdůvodně prohlížen, mučen, podroben krutému, nelidskému nebo ponižujícímu trestu
    \item do obydlí člověka není dovoleno vstoupit nikomu bez souhlasu, prohlídka je přípustná jen pro účely trestního řízení
\end{itemize}

\subsubsection{Právo na osobní svobody}
\begin{itemize}
    \item nikdo nesmí být stíhán nebo zadržen než z důvodu a způsobem, který stanoví zákon
\end{itemize}

\subsubsection{Zákaz nucených prací nebo služeb}
\begin{itemize}
    \item výjimky: práce v trestu odnětí svobody, vojenská služba, služba při živelných pohromách
\end{itemize}

\subsubsection{Právo vlastnické}
\begin{itemize}
    \item vyvlastnění či omezení vlastnického práva je možné jen na základě zákona
\end{itemize}

\subsubsection{Další práva osobního charakteru}
\begin{itemize}
    \item právo na lidskou důstojnost, osobní čest, dobrou pověst a jméno, právo na ochranu před zneužíváním osobních údajů
    \item svoboda pohybu a pobytu
    \item svoboda myšlení, náboženského vyznání, umělecké tvorby...
\end{itemize}

\subsubsection{Svoboda projevu}
\begin{itemize}
    \item možnost projevovat své názory na veřejnosti, a to slovem, tiskem, obrazem, filmem atd.
\end{itemize}

\subsubsection{Svoboda sdružovací}
\begin{itemize}
    \item právo svobodně se sdružovat ve spolcích, společnostech, politických stranách, hnutích...
\end{itemize}

\subsubsection{Právo shromažďovací}
\begin{itemize}
    \item demonstrace
\end{itemize}

\subsubsection{Právo petiční}
\begin{itemize}
    \item např. na vznik strany, kandidát na prezidenta (50000 podpisů)
\end{itemize}

\subsubsection{Právo volební}
\begin{itemize}
    \item aktivní a pasivní
\end{itemize}

\subsubsection{Práva národnostních a etnických menšin}
\begin{itemize}
    \item příslušnost ke kterékoli národnostní nebo etnické menšině nesmí být nikomu na újmu
\end{itemize}

\subsubsection{Právo na svobodnou volbu povolání, podnikání a další práva}
\begin{itemize}
    \item každý se může rozhodnout, zda chce pracovat a jakou práci by chtěl vykonávat, či zda chce podnikat
\end{itemize}

\subsubsection{Právo na zabezpečení}
\begin{itemize}
    \item každý má právo na přiměřené hmotné zabezpečení ve stáří, při nezpůsobilosti k práci a při ztrátě živitele
\end{itemize} 

\subsubsection{Ochrana rodiny}
\begin{itemize}
    \item děti narozené v manželství i mimo mají stejná práva
    \item rodiče, kteří pečují o děti, mají právo na pomoc státu
\end{itemize} 

\subsubsection{Právo na vzdělání}
\begin{itemize}
    \item školní docházka je po určitou dobu povinná
    \item vzdělání na státních základních a středních školách je bezplatné
\end{itemize} 

\subsection{Dělba moci}
\begin{itemize}
    \item \textbf{Moc zákonodárná} - oprávnění vydávat zákony. V České republice je to Parlament, který se skládá ze
    dvou komor: Poslanecké sněmovny a Senátu.
    \item \textbf{Moc výkonná} - Exekutiva; Je pověřená ke každodennímu řízení státu, státní správy. Výkonná moc obvykle
    náleží vládě, či čelnímu představiteli země.
    \item \textbf{Moc soudní} - Justice; Soudnictví vykonávají specifikované státní orgány, kterými jsou nezávislé soudy.
    Ty zákonem stanoveným způsobem zajišťují v občanskoprávním řízení ochranu subjektivních práv, 
    v trestním řízení rozhodují o vině a trestu za trestné činy, ve správním soudnictví přezkoumávají akty orgánů veřejné správy a v ústavním soudnictví rozhodují o souladu právních předpisů i rozhodnutí s ústavou, případně rozhodují i o dalších věcech,
    které jim jsou zákonem svěřeny.
\end{itemize}

\newpage

\subsection{Legislativní proces}
\begin{itemize}
    \item proces tvorby práva
    \item začátek legislativního procesu je dán prvotním návrhem nějaké právní normy = \textbf{legislativní záměr}
    \begin{itemize}
        \item může podat někdo se zákonodárnou iniciativou = vláda, poslanec, skupina poslanců, senát, kraj
        \item zpráva pojednávající o tom, že by měla být schválena nová právní norma s odůvodněním proč,
        co a jaké instituce mají zřídit a kolik to bude stát
    \end{itemize}
    \item po podrobení legislativního záměru oponentuře se zpracuje paragrafové znění - \textbf{návrh zákona}    
    \begin{itemize}
        \item návrh zákona opět projde připomínkovým řízením legislativní rady vlády a poté je v konečné úpravě
        odeslán předsedovi Poslanecké sněmovny (Jan Hamáček), Parlamentu ČR, poté se návrh rozešle všem poslancům
    \end{itemize}
\end{itemize}

\subsubsection{Poslanecká sněmovna}
Rozhodující fáze zákonodárného procesu je projednání návrhu v Poslanecké sněmovně, tzv. ,,čtení''

\begin{enumerate}
    \item čtení - předložení zákona, hlasování o něm. Výsledkem může být zamítnutí návrhu,
    postoupení výborům s následným druhým čtením, vrácení předkladateli.
    \item čtení - jednání ve výborech, podrobná rozprava v Poslanecké sněmovně, předkládají se pozměňovací návrhy
    a hlasuje se o zamítnutí návrhu, vrácení výboru nebo postoupení ke 3. čtení
    \item čtení - pouze odstranění případných drobných vad, případně gramatických chyb, končí hlasování -
    je-li návrh přijat, jde do Senátu
\end{enumerate}

\subsubsection{Senát}
\begin{itemize}
    \item předseda Senátu rozešle návrh senátorům a příslušným výborům, Senát návrh schválí, zamítne 
    nebo vrátí Poslanecké sněmovně s pozměňovacími návrhy
    \item o návrhu Senátem zamítnutého zákona hlasuje Sněmovna znovu, jestliže se zákonem vysloví souhlas
    nadpoloviční většina všech poslanců (101), je zamítnutí senátem tzv. \textbf{přehlasováno} a platí původní znění
\end{itemize}

\subsubsection{Prezident republiky}
Schválený zákon je postoupen prezidentovi republiky, který do 15 dnů může využít suspenzivního veta, které lze
přehlasovat nadpoloviční většinou všech poslanců
\begin{itemize}
    \item Vyhlášení 
    \begin{itemize}
        \item přijaté zákony podepisuje předseda poslanecké sněmovny, prezident a a nakonec předseda vlády
        \item k platnosti zákona je třeba, aby byl vyhlášen = publikován ve Sbírce zákonů, uveřejněním ve
        Sbírce zákonů se stává \textbf{platným}, \textbf{účinným} je teprve datem uvedeným v zákoně
    \end{itemize}
\end{itemize}

\subsection{Orgány moci zákonodárné}

\subsubsection{Parlament ČR}
\begin{itemize}
    \item imunita poslanců a senátorů
    \begin{itemize}
        \item nelze je stíhat pro hlasování, za projevy v Poslanecké sněmovně, v Senátu
        \item nelze je stíhat za přestupky a trestné činy bez souhlasu komory, jejímiž jsou členy
        \item lze je zadržet, jen byli-li dopadeni při spáchání trestného činu, nebo bezprostředně poté
    \end{itemize}
\end{itemize}

\subsubsection{Poslanecká sněmovna}
\begin{itemize}
    \item volby poslanců
    \begin{itemize}
        \item 200 poslanců, starších 21 let, občanů ČR, voleno na základě poměrného zastoupení na 4 roky
        \item volby se konají ve lhůtě počínající 30. dnem před uplynutím volebního období a končí dnem
        jeho uplynutí
        \item byla-li Poslanecká sněmovna rozpuštěna, konají se volby do 60 dnů po jejím rozpuštění
    \end{itemize}
    \item rozpuštění Poslanecké sněmovny
    \begin{itemize}
        \item provádí prezident, jestliže např. nebyla po dobu delší než tři měsíce způsobilá se usnášet
        \item dojde-li k rozpuštění Poslanecké sněmovny, přísluší Senátu přijímat zákonná opatření ve věcech,
        které nesnesou odkladu
    \end{itemize}
    \item činnosti Poslanecké sněmovny
    \begin{itemize}
        \item projednávání a schvalování zákonů
        \item kontrola činnosti vlády ve formě interpelací
    \end{itemize}
\end{itemize}

\subsubsection{Senát}
\begin{itemize}
    \item volby senátorů
    \begin{itemize}
        \item 81 senátorů, starších 40 let, občanů ČR, voleno na základě většinového systému na 6 let
        \item každé 2 roky se volí nová třetina senátorů
    \end{itemize}
    \item činnosti senátu
    \begin{itemize}
        \item projednávání a schvalování zákonů schválených Poslaneckou sněmovnou
    \end{itemize}
\end{itemize}

\subsection{Orgány moci výkonné v ČR}

\subsubsection{Prezident ČR}
\begin{itemize}
    \item volba prezidenta
    \begin{itemize}
        \item prezidentem může být zvolen státní občan ČR, který dovršil věku 40 let a má právo volit
        \item volební období je 5 let, nikdo nemůže být zvolen více než dvakrát za sebou
        \item navrhnout prezidentského kandidáta může:
        \begin{itemize}
            \item každý občan starší 18 let  minimálně 50000 podpisy
            \item minimálně 20 poslanců Parlamentu ČR
            \item minimálně 10 senátorů ČR
        \end{itemize}
        \item prezident se volí přímo
        \begin{itemize}
            \item v prvním kole prezidentských voleb zvítězí kandidát, který obdrží nadpoloviční počet platných hlasů
            od oprávněných voličů
            \item není-li zvolen prezident v prvním kole nadpoloviční většinou platných hlasů, koná se za 14 dní 
            po začátku prvního kola volby druhé kolo volby, do kterého postupují 2 nejúspěšnější kandidáti z 
            prvního kola volby, v druhém kole je zvolen prezidentem kandidát, který v druhém kole dosáhl nejvyššího
            počtu hlasů
        \end{itemize}
    \end{itemize}
    \item pravomoci prezidenta
    \begin{itemize}
        \item jmenuje a odvolává předsedu a členy vlády a přijímá jejich demisi, odvolává vládu a přijímá její demisi
        \item pověřuje vládu, jejíž demisi přijal nebo kterou odvolal, vykonáváním funkcí až do jmenování nové vlády
        \item svolává zasedání Poslanecké sněmovny a rozpouští Poslaneckou sněmovnu
        \item jmenuje soudce Ústavního soudu, jeho předsedu a místopředsedy
        \item jmenuje ze soudců předsedu a místopředsedy Nejvyššího soudu
        \item odpouští a zmírňuje tresty uložené soudem (agraciace) a zahlazuje odsouzení (rehabilitace) -
        nemůže na nikoho přenést
        \item má právo vrátit Parlamentu zákon (mimo ústavního)
        \item podepisuje zákony
        \item jmenuje prezidenta a víceprezidenta Nejvyššího kontrolního úřadu
        \item jmenuje členy Bankovní rady ČNB
    \end{itemize}
    
    \item další pravomoci prezidenta (\textit{vyžadují spolupodpis předsedy vlády nebo člena vlády})
    \begin{itemize}
        \item zastupuje stát navenek
        \item sjednává a ratifikuje mezinárodní smlouvy (sjednávat za něj může i vláda)
        \item je vrchním velitelem ozbrojených sil
        \item vyhlašuje volby do Poslanecké sněmovny a Senátu
        \item pověřuje a odvolává vedoucí zastupitelských misí
        \item jmenuje a povyšuje generály
        \item uděluje a propůjčuje státní vyznamenání
        \item jmenuje soudce
        \item nařizuje, aby se trestní stíhání nezahajovalo, a bylo-li zahájeno, aby se v něm nepokračovalo
        \item má právo udělovat amnestii
    \end{itemize}
    
    \item práva prezidenta
    \begin{itemize}
        \item účastnit se schůzí obou komor Parlamentu, jejich výborů i komisí a dostat slovo, kdykoliv o to požádá
        \item účastnit se schůzí vlády, vyžadovat od ní a jejích členů zprávy
        \item projednávat s vládou a jejími členy otázky, které patří do jejich působnosti
        \item prezidenta ČR nelze po doby výkonu funkce zadržet, trestně stíhat ani ho stíhat pro přestupek a jiný
        právní delikt
        \item může být stíhán pouze pro velezradu před Ústavním soudem a na základě žaloby Senátem
    \end{itemize}
\end{itemize}

\subsubsection{Vláda ČR}
\begin{itemize}
    \item vrcholný orgán výkonné moci v ČR
    \item skládá se z předsedy vlády, místopředsedů vlády a ministrů
    \item premiéra jmenuje prezident a na jeho doporučení pak ministry vlády
    \item je odpovědná Poslanecké sněmovně
    \item vyslovení důvěry vládě
    \begin{itemize}
        \item do 30 dní po jmenování
        \item pokud vláda důvěru nezíská, jmenuje prezident ČR novou vládu
        \item jestliže ani tato vláda nezíská důvěru, sněmovna se rozpustí a konají se nové volby
        \item vláda může předložit Poslanecké sněmovně žádost o vyslovení důvěry v průběhu volebního období
        \item Poslanecká sněmovna může kdykoliv vyslovit vládě nedůvěru, pokud návrh podá alespoň 50 poslanců, k přijetí
        návrhu je potřeba souhlas nadpoloviční většiny všech poslanců
    \end{itemize}
    \item demise vlády
    \begin{itemize}
        \item jako celkem podává demisi vláda, které Poslanecká sněmovna vyslovila nedůvěru nebo jejíž žádost o 
        důvěru byla zamítnuta
        \item individuálně podávají demisi členové vlády do rukou prezidenta ČR
        \item vždy po ustavující schůzi nově zvolené Poslanecké sněmovny
    \end{itemize}
    \item činnost vlády
    \begin{itemize}
        \item vláda rozhoduje ve sboru, k přijetí usnesení vlády je potřeba souhlasu nadpoloviční většiny jejích členů
        \item oprávněna vydávat nařízení k provedení zákona
    \end{itemize}
\end{itemize}

\newpage

\section{Právní norma \\ \small Klasická struktura právní normy, druhy právních norem}

\subsection{Právní norma}
Obecně závazné pravidlo chování, stanovené (nebo uznané) státem a vynucované státní mocí

\subsection{Struktura právních norem}
Vnitřní forma, v níž je obsah určitým způsobem organizován

\subsubsection{Klasická struktura}
\begin{itemize}
    \item \textbf{hypotéza} - podmínky realizace dispozice
    \item \textbf{dispozice} - vlastní pravidlo chování. Z hlediska normotvůrce $ \rightarow $ příkaz, zákaz, dovolení.
    Z pohledu adresáta $ \rightarrow $ oprávnění, povinnost něco vykonat, něčeho se držet
    \item \textbf{sankční hypotéza} - stanoví, zda se v případě porušení chování v dispozici použije, 
    či nepoužije dále stanovená sankce
    \item \textbf{sankce} - stanoví důsledky porušení pravidel dispozice
\end{itemize}

\subsubsection{Bez klasické struktury}
\begin{itemize}
    \item \textbf{normy blanketové} - odkazují na jinou normu (obecně určenou), 
    např. ,,pokud v zákoně není stanoveno jinak''
    \item \textbf{normy odkazovací} - odkazují na jinou, jmenovitě stanovenou normu, např. §63
    \item \textbf{normy kolizní} - určují rozhodné právo pro úpravu vztahů s mezinárodním prvkem. Určují, zda se 
    použije např. právní řád ČR, nebo jiné země.
\end{itemize}

\subsection{Druhy právních norem}
\begin{itemize}
    \item \textbf{kogentní} - nelze se od nich odchýlit, zavazují své adresáty bezvýhradně a musí být použity vždy.
    Typicky se jedná o veřejné právo
    \item \textbf{dispozitivní} - lze se od nich dohodou stran platně odchýlit, ale jsou omezené kogentními normami.
    Typicky se jedná o soukromé právo
\end{itemize}

\subsection{Dělení z hlediska jazykového vyjádření dispozice}
\begin{itemize}
    \item opravňující
    \item zavazující
    \item přikazující
    \item zakazující
\end{itemize}

Znaky právních norem jsou obecnost, formální účinnost, závaznost, vynutitelnost

\subsection{Platnost a působnost právních norem}

\subsubsection{Platnost právních norem}
\begin{itemize}
    \item platnost právní normy souvisí s její normativní existencí - je součástí daného právního řádu za 
    splnění všech požadovaných náležitostí $ \rightarow $ musí být vydána státním orgánem, který je k jejímu
    vydání kompetentní, musí mít formu některého pramene práva a řádně vyhlášena
    \item \textbf{právní předpisy v ČR nabývají platnosti dnem vyhlášení ve Sbírce zákonů}
    \item to, že je právní norma platná, znamená, že byla \textbf{publikována}, ale ještě se jí
    nemusíme řídit
\end{itemize}

\subsubsection{Účinnost právních norem}
\begin{itemize}
    \item právní norma je účinná, jakmile může způsobit právní následky s jejím přijetím spojené, tedy
    může být aplikována na ní upravené právní vztahy
    \item okamžik platnosti a účinnosti nemusí splývat v jeden den
    \item podmínkou účinnosti je platnost
    \item den nabytí účinnosti bývá zpravidla určen, pokud tomu tak není, tak nabývá účinnosti 15. den
    od nabytí platnosti
\end{itemize}

\subsubsection{Působnost právních norem}
\begin{itemize}
    \item působnost právní normy vymezuje rozsah aplikace a realizace právní normy
    \item $ \rightarrow $ na jaký okruh případů se dá právní norma aplikovat se zřetelem k času (časová působnost),
    místu (prostorová působnost) nebo předmětu úpravy (působnost osobní a věcná)
    \item působnosti
    \begin{itemize}
        \item časová
        \begin{itemize}
            \item vymezuje dobu, po kterou právní norma existuje, resp. je součástí právního řádu
            \item otázka časové působnosti souvisí s platností a účinností právního předpisu s retroaktivitou
            \item derogační klauzule je výslovné ustanovení k zániku platnosti normy,
            tedy jejího vyřazení z právního řádu
            \begin{itemize}
                \item generální derogační klauzule - ruší všechny právní normy, jež odporují novému normativnímu aktu,
                nese s sebou nutnost zjisti, které normy byly zrušeny.
                \item taxativní derogační klauzule - stanoví taxativním (úplným) výčtem, která ustanovení se ruší,
                žádné pochyby
            \end{itemize}
            \item zpravidla normy nepůsobí zpětně (retroaktivita), až na výjimky
            \begin{itemize}
                \item pravá retroaktivita - účinnost právního předpisu začne dříve než jeho platnost. 
                Vytvoří se tedy právní fikce, že právní předpis byl účinný již v době, kdy ještě neexistoval - např. 
                povoleno v trestním právu, je-li to pro pachatele výhodnější
                \item nepravá zpětná účinnost - vznik právního vztahu se určuje podle normy účinné v době vzniku, ale
                obsah vztahu se určuje podle normy nové
            \end{itemize}
        \end{itemize}
        
        \item prostorová - stanovení rozsahu, v jakém se norma použije se zřetelem k místu
        \begin{itemize}
            \item trestní zákon se vztahuje na trestné činy spáchané na území ČR, bez ohledu na to, zda je spáchal
            občan ČR či cizinec (princip teritoriality)
        \end{itemize}
        
        \item osobní - úzce spojena s prostorovou působností, odpovídá na otázku, pro které osoby právní norma platí
        \begin{itemize}
            \item nejčastěji se vtahuje na osoby (subjetky) na území ČR
        \end{itemize}
        
        \item věcná - vymezení okruhu společenských vztahů, na které se právní norma vztahuje
    \end{itemize}
\end{itemize}

\newpage

\section{Právní vztahy \\ \small{Právní poměry - pojem, druhy, předpoklady a prvky}}

\subsection{Právní vztahy}
 Vztahy mezi lidmi či jejich organizovanými kolektivy, v nichž tito vystupují jako nositelé oprávnění
 a povinností stanovených právními normami
 
\subsection{Druhy právních vztahů}
\subsubsection{Podle právních odvětví}
\begin{itemize}
    \item pracovněprávní, občanskoprávní, správně právní
\end{itemize}

\subsubsection{Podle vzájemného postavení subjektů}
\begin{itemize}
    \item \textbf{horizontální} - mají vzájemně rovné postavení
    \item \textbf{vertikální} - určitý subjekt má mocenské postavení
\end{itemize}

\subsection{Předpoklady}
K tomu, aby právní vztah mohl vzniknout, je třeba, aby byly splněny dva předpoklady, existence
právní normy, která takový vztah upravuje a právní skutečnost, které způsobují vznik, změnu nebo zánik tohoto 
právního poměru.

\subsubsection{Právní normy}
\subsubsection{Právní skutečnosti}
\begin{itemize}
    \item rozlišují se aprobované (právní chování) a reprobované (protiprávní chování)
    \item jsou projevem vůle subjektu - právní i protiprávní jednání
    \item skutečnosti, které nejsou projevem vůle subjektu - právní události, protiprávní stavy
\end{itemize}

\subsection{Prvky právních vztahů}
\begin{itemize}
    \item subjekty (účastníci) - ti, jimž v právním vztahu vznikají práva a povinnosti (FO, PO, stát a státní orgány)
    \item obsah - souhrn oprávnění a povinností vyplývajících subjektům z daného právního vztahu.
    \begin{itemize}
        \item oprávnění - míra možnosti chování subjektu, možnost chovat se určitým způsobem je chráněna a zaručena právem
        \item povinnost - účastník se musí chovat tak, jak mu to právní norma přikazuje, jinak porušuje právní normu a 
        vystavuje se možnosti uplatnění státního donucení (sankce)
    \end{itemize}
    \item objekt - to, k čemu směřují vzájemná práva a povinnosti subjektů (= cíl právního vztahu)
    \begin{itemize}
        \item přímý objekt právního vztahu - dát, konat, zdržet, strpět
        \item nepřímý objekt právního vztahu - objektem je to, k čemu chování pracovníků směřuje (hmotné a nehmotné
        statky, práva, hodnoty lidské osobnosti)
    \end{itemize}
\end{itemize}

\section{Právní úkony, právní jednání \\ \small{Pojem, druhy, náležitosti, relativní neúčinnost}}

\subsection{Právní jednání}
\begin{itemize}
    \item projev vůle subjektu směřující ke způsobení právních následků
    \item \textbf{projevy vůle}
    \begin{itemize}
        \item \textbf{výslovně} - písemně, ústně či určitými symboly
        \item \textbf{konkludentně} - projev vůle učiněný jiným způsobem než výslovně (tedy ne ústně nebo písemně)
        přičemž účastník právního úkonu takovým jednáním, jako je např. kývnutí hlavou, potřesení rukou,
        v obchodních vztazích také dodáním zboří, vyjádří svou vůli
        \item \textbf{tacitně} - projevem vůle je nečinnost
    \end{itemize}
    \item \textbf{Obligatorně písemná forma} - práva k nemovitostem a podpis na téže listině u více osob.
\end{itemize}

\subsection{Druhy právních úkonů}
\begin{itemize}
    \item \textbf{jednostranná, dvoustranná, vícestranná} - jednostranné úkony jsou projevy vůle jednoho subjektu
    (např. závěť), dvoustranné a vícestranné spočívají ve shodě projevů vůle (konsensu) více subjektů
    \item \textbf{adresovaná, neadresovaná} - musí být někomu adresovány (reklamace), či nikoliv (veřejný příslib)
    \item \textbf{úplatné, neúplatné} - má-li strana, která úkon činí, obdržet na jeho základě majetkovou hodnotu
    (většina úkonů) či nikoliv (darovací smlouva)
    \item \textbf{pojmenovaná, nepojmenovaná} - zda právní normy tyto úkony výslovně upravují čí nikoliv
    \item \textbf{právní odvětví} - dělení z hlediska systému práva (např. pracovněprávní úkony, občanskoprávní úkony)
\end{itemize}

\subsection{Náležitosti právních jednání}
\begin{itemize}
    \item \textbf{náležitost subjektu} - předpokladem platnosti právního jednání je právní subjektivita 
    (pokud koná nesvéprávný, tak je úkon neplatný)
    \item \textbf{náležitost vůle} - vůle jednajícího musí být svobodná, vážná a prostá omylu (nesmí k ní dojít
    v důsledku omylu)
    \item \textbf{náležitost projevu vůle} - srozumitelnost, určitost
    
\newpage

    \item \textbf{náležitosti obsahu}
    \begin{itemize}
        \item \textbf{podstatné složky} - např. v pracovní smlouvě je to druh vykonávané práce
        \item \textbf{pravidelné složky} - např. stanovení odměny za vykonávanou práci
        \item \textbf{nahodilé složky} - např. sjednání kratší pracovní doby
        \item \textbf{předmět právního jednání} - plnění právního jednání musí být možné a dovolené
        \item \textbf{nedovolenost právního jednání} - pokud je jednání v rozporu se zákonem
    \end{itemize}
\end{itemize}

\subsubsection{Platnost právního jednání}
Jednání je pro toho, kdo jej učinil, závazné = nemůže být libovolně odvoláno, změněno

\subsubsection{Účinnost právního jednání}
Nastanou právní důsledky, ke kterým jednání směřovalo

\newpage

\section{Vady právního jednání \\ \small{Zdánlivost, neplatnost, relativní neúčinnost}}

\subsection{Neplatnost právního jednání}
Právní jednání je neplatné, nesplňuje-li právní úkon podstatné náležitosti, je vadný - nejčastěji neplatný

\subsubsection{Absolutní neplatnost}
Právní úkon nevyvolává předpokládané právní následky, na jeho základě nelze vyžadovat plnění. 
Absolutně neplatné právní jednání nastává, pokud:
\begin{itemize}
    \item se příčí dobrým mravům
    \item odporuje zákonu a zjevně narušuje veřejný pořádek
    \item zavazuje k plnění od počátku k nemožnému
\end{itemize}

\subsubsection{Relativní neplatnost}
Právní úkon je platný do doby, než se oprávněná osoba dovolá, do té doby musí být plněno

\subsubsection{Odporovatelnost - relativní neúčinnost} 
Věřitel se může domáhat vůči své osobně neúčinnosti právních úkonů, kterými se dlužník zbavuje svého majetku,
aby nemusel plnit závazky vůči věřiteli

\subsubsection{Konvalidace}
Např. organizace dodatečně schválí úkon, který jejím jménem učinil někdo, kdo k tomu nebyl oprávněn

\subsubsection{Zdánlivé jednání}
Na rozdíl od jednání neplatného rozlišuje právní úprava situace, kdy vůbec nejsou splněny náležitosti pro to.
aby se jednalo o právní jednání. 
Jedná se o případy:

\begin{itemize}
    \item chybí vůle subjektu
    \item nebyla projevena vážná vůle subjektu
    \item nelze pro neurčitost nebo nesrozumitelnost zjistit jeho obsah ani výkladem
\end{itemize}

Jednání je zdánlivé i v případech, kdy je jednání v rozporu s důležitými právními či morálními zásadami, nebo je
v rozporu s ustanovením na ochranu určitých osob.

\newpage

\section{Právní události, lhůty a doby v právu, protiprávní stavy, právní domněnky a fikce}

\subsection{Právní události}
Skutečnosti, které nastávají nezávisle na vůli jakékoliv osoby a na něž se váže vznik, změna nebo zánik právního vztahu,
ale nebyly vyvolány chováním osob jimi dotčených.
Jedná se např. o plynutí času, narození, smrt, dosažení určitého věku, katastrofické jevy atd.

\subsubsection{Právní poměr}
\begin{itemize}
    \item vzniká
    \item mění se 
    \item zaniká 
\end{itemize}

\subsection{Lhůta a doba v právu}

\subsubsection{Lhůta}
Časový úsek, určený právním předpisem, případně rozhodnutím příslušného orgánu, nebo právním jednáním k aktivnímu
uplatnění práva u druhé strany.

\subsubsection{Doba}
Časový úsek, po jehož uplynutí zaniká právo nebo povinnost bez dalšího jednání nebo projevu vůle.

\subsubsection{Marné uplynutí doby}
Situaci, když nedojde k aktivnímu uplatnění práva, které je omezeno na určitou lhůtu.

\subsubsection{Jiné časové úseky}
Časové úseky, které jsou určené přibližně či neurčitě (např. ,,bez zbytečného odkladu'')

\subsubsection{Dělení lhůt}
\begin{itemize}
    \item podle situace, která nastane při marném uplynutí lhůty
    \begin{itemize}
        \item \textbf{lhůty prekluzivní} - její marné uplynutí má za následek zánik práv. ten kdo by plnil po
        uplynutí prekluzivní lhůty, plnil by bez právního důvodu
        \item \textbf{lhůty promlčecí} - jejich marným uplynutím právo nezaniká, nastává však promlčení
        (vznik možnosti povinného subjektu uplatnit námitku promlčení)
    \end{itemize}
    
    \item podle okolností rozhodujících pro začátek běhu lhůty
    \begin{itemize}
        \item \textbf{objektivní lhůty} - rozhodující je, kdy se daná skutečnost stala
        \item \textbf{subjektivní lhůty} - rozhodující je, kdy se subjekt o skutečnosti dozvěděl
    \end{itemize}
    
    \item hmotněprávní a procesní
    \begin{itemize}
        \item \textbf{lhůty a doby hmotněprávní} - význam v oblasti upravované hmotným právem
        \item \textbf{lhůty a doby procesní} - význam z hlediska některého procesu aplikace práva
    \end{itemize}
\end{itemize}

\subsubsection{Stavění lhůty/doby}
Hovoří se o případě, že lhůta nebo doba v důsledku určité události nebo situace přestává běžet.
Když tato událost nebo situace pomine, pokračuje lhůta nebo doba tam, kde se zastavila.

\subsubsection{Přerušení (přetržení) lhůty/doby}
Událost nevyvolá jen zastavení běhu lhůty nebo doby, ale vrátí ji na počátek, takže poté, co událost skončí, 
se rozběhne v celé své délce

V některých případech lhůta ani nezačne běžet, např. manželům nezačne ani neběží lhůta, dokud manželství trvá

\subsection{Protiprávní stavy}
Jedná se o výsledek nezaviněného chování nebo události, které jsou v rozporu s právem.
Právní norma ukládá někomu povinnost takovýto stav napravit, nebo odstranit, a to někomu, kdo
stav svým jednáním nezavinil.
Organizace odpovídá za škodu vzniklou pracovníkům při plnění pracovních úkolů pracovním úrazem nebo
nemoci z povolí i tehdy, jestliže škodu nezavinila

\subsection{Právní domněnky a fikce}

\subsubsection{Vyvratitelná právní domněnka}
Platí, dokud se neprokáže opak - presumpce neviny

\subsubsection{Nevyvratitelná právní domněnka}
Nelze ani důkazy o opaku vyvrátit

\subsubsection{Právní fikce}
Jedná se o právní konstrukci, která spojuje právní následky se skutečností, která ještě nenastala nebo reálně
neexistuje. Např. na počaté dítě se hledá jako na již narozené.

\newpage

\section{Osoby v právním smyslu \\ \small{Právní osobnost, svéprávnost, přiznání, omezení}}

\subsection{Fyzická osoba}
Každý člověk má vrozená, již samotným rozumem a citem poznatelná přirozená práva, a tudíž se považuje za osobu.

\subsection{Právnická osoba}
Právnická osoba je organizovaný útvar, o kterém zákon stanoví, že má právní osobnost, nebo jehož právní osobnost 
zákon uzná.
Aby fyzické osoby a právnické osoby mohli být subjektem právního poměru, musí mít svou osobnost a svéprávnost.

\subsection{Osobnost}
Jedná se o způsobilost být účastníkem právních vztahů (mít práva a povinnosti v mezích právního řádu)

\subsubsection{Osobnost fyzických osob}
Právní osobnost vzniká narozením a končí smrtí. Náleží ji vrozená práva seznatelná samotným rozumem a citem. 
Tato právo jsou nezcizitelná a nelze se jich vzdát. Smrt se prokazuje úmrtním listem, v mimořádných případech
důkazem smrti, nebo prohlášením za mrtvého.

\subsubsection{Osobnost právnických osob}
Právní osobnost má od svého vzniku do svého zániku. Právnická osoba vzniká dnem zápisu do obchodního rejstříku a 
zaniká výmazem z něj, případně likvidací.

\subsection{Svéprávnost}
Jedná se o možnost právně jednat.

\subsubsection{Svéprávnost fyzických osob}
Osoba dosahuje plné svéprávnosti zletilostí (dosažení věku 18 let), případně dřív a to buď emancipací nebo sňatkem.
Svéprávnosti lze člověka omezit, nikoliv zbavit a to pouze rozhodnutím soudu z důvodu vážné duševní nemoci a jen 
na určitý čas.

Způsobilost k protiprávním jednání je způsobilost nést právní odpovědnost za svá jednání. Každé fyzické osobě
je uznána od dosažení věku 15 let, případně při příčetnosti v době spáchání trestného činu (význam pro trestní právo)

\subsubsection{Svéprávnost právnických osob}
Právnické osoby nemohou mít svéprávnost, mají pouze osobnost. Z tohoto důvodu právnickou osobnost zastupuje
statutární orgán. Ten je stanoven v zákoně, případně je určen při zakladatelském právním jednání.

\newpage

\section{Zastoupení \\ \small{Pojem, druhy, smluvní zastoupení a plná moc, zákonné zastoupení, opatrovnictví}}

\subsection{Zastoupení}
Jedná se o nepřímé jednání dané fyzické nebo právnické osoby na základě smlouvy nebo zákona. 
Ze zastoupení obecně vznikají práva a povinnosti přímo zastoupenému, ale není-li zřejmé, že někdo
jedná za jiného, platí, že jedná vlastním jménem. 

\begin{itemize}
    \item zastoupit jiného nemůže ten, jehož zájmy jsou v rozporu se zájmy zastoupeného
    \item zástupce jedná osobně, ledaže je nutná potřeba nebo je se zastoupeným ujednáno, že pověří dalšího zástupce,
    pak ovšem odpovídá za řádný výběr takové osoby
\end{itemize}

\subsection{Druhy zastoupení}

\subsubsection{Smluvní zastoupení}
Vzniká na základě \textbf{smlouvy o zastoupení} dohodou \textbf{zmocnitele} a \textbf{zmocněnce}

\begin{itemize}
    \item \textbf{plná moc} - zmocnitel uvede rozsah zmocnění v plné moci, jež je ,,legitimací'' o tomto zmocnění a 
    jeho rozsahu pro třetí osobu. Plná moc je jen jednostranné potvrzení, že zastoupení bylo ujednáno a vzniklo
    \begin{itemize}
        \item \textbf{speciální plná moc} - týká se určitého jednotlivého právního jednání (např. uzavření smlouvy 
        se třetí osobou), nebo právního jednání v rámci celého případu (např. zastoupení advokátem ve sporu)
        \item \textbf{generální plná moc} - ke všem jednáním
    \end{itemize}
    \item vedle plné moci je tedy závazek mezi oběma stranami upraven právě ve smlouvě - ta se ale obvykle třetí osobně
    nepředkládá, tj. z tohoto důvodu se vystavuje plná moc
    \item plná moc se uděluje \textbf{písemně}, netýká-li se zastoupení jen určitého právního jednání
    
    \item \textbf{prokura} - zvláštní případ smluvního zastoupení podnikatele (velmi široká ,,podnikatelská plná moc'')
\end{itemize}

\subsubsection{Zákonné zastoupení}
Vzniká přímo ze zákona, jedná se o situace, kde je zastoupení nutné pro ochranu zájmů zastoupeného 
- například z důvodu nedostatku věku nebo nesvéprávnosti
(např. zastoupení nezletilého rodičem, vzájemné zastoupení manželů)

\subsubsection{Zastoupení na základě rozhodnutí soudu}
Jedná se o jmenování opatrovníka soudem, kde soud určí rozsah práv a povinností opatrovníka.
Soud jmenuje opatrovníka člověku, je-li to potřeba k ochraně jeho zájmů (např. nesvéprávnému člověku), 
může to být např. osoba blízka.
Nově lze ustanovit opatrovníka i právnické osobě v případě
\begin{itemize}
    \item nemá-li statutární orgán právnické osoby dostatečný počet členů k rozhodování
    \item jsou-li zájmy člena statutárního orgánu v rozporu se zájmy právnické osoby, pokud daná právnická
    osoba nemá jiného člena statutárního orgánu, který by ji byl schopen zastupovat. Tzv. zájmová kolize
    \item je-li to potřebné ke správě jejích záležitostí nebo k ochraně jejích práv
\end{itemize}

Opatrovník má obdobná práva a povinnosti jako člen statutárního orgánu, při opatrovnictví je stanovena
opatrovnická rada, která dohlíží na činnost opatrovníka.

\newpage

\section{Právnické osoby \\ \small{pojem, druhy, znaky, vznik, orgány, jednání, spolky, fundace, ústav}}

\subsection{Právnické osoby}
Jedná se o společenské útvary (organizace, obchodní společnosti apod.), jimž právní řád přiznává právní subjektivitu,
tj. způsobilost k právům a povinnostem - způsobilost být účastníkem vztahů

\subsection{Orgány právnických osob}

\subsubsection{Dělení podle počtu členů} 
\begin{itemize}
    \item individuální
    \item kolektivní
\end{itemize}

\subsubsection{Dělení podle činnosti}
\begin{itemize} 
    \item nejvyšší
    \item statutární - jedná za právnickou osobu ve všech záležitostech
    \item kontrolní
\end{itemize}

\subsection{Druhy právnických osob}

\subsubsection{Fundace}
Fundace je právnická osoba založená majetkem vytvořeným k určitému účelu. Její činnost se váže na účel, za kterým
byla založena. Fundace vytváří osamostatněný majetek, nemá společníky (nahrazuje je vůle zakladatele).
U fundací je obligatorní existence statutárního a kontrolního orgánu. Kromě těchto orgánů není zakladatel
nijak omezen a může zřídit funkci výkonného ředitele, správce, poradce, atd.

\subsubsection{Nadace}
Nadace je sdružení majetku za účelem dosahování společensky nebo hospodářsky užitečných cílů, např.
nadace Dobrý anděl, nadace A. Nobela (Nobelovy ceny). 
\\\\ % TODO: fix this ugly shit
Nadace se zakládá nadační listinou, vzniká zápisem do veřejného rejstříku. Obligatorně musí mít statutární orgán, 
kterým je správní rada s nejméně 3 členy a kontrolní orgán, kterým je dozorčí rada s nejméně 3 členy, nebo revizor.
\\\\
Nadace mohou podnikat pouze na vedlejší činnost.

\newpage 

\subsubsection{Korporace}
Korporace jsou nejstarším druhem právnické osoby, je to právnická osoba tvořena společenstvím osob.
\begin{itemize}
    \item \textbf{spolky}
    \begin{itemize}
        \item základní typ korporace
        \item má alespoň 3 členy
        \item hlavní účel nesmí být podnikání, podnikat může pouze jako vedlejší činnost
        \item zákon nepředvídá vkladovou povinnost
        \item musí mít vnitřní orgán, který bude určeno jako statutární orgán (výbor, předseda, ...)
        \item musí mít také nejyšší orgán, který ale může být shodný se statutárním orgánem
    \end{itemize}
    \item \textbf{obchodní korporace}
    \begin{itemize}
        \item obchodní společnosti
        \item družstva
    \end{itemize}
    \item \textbf{ústavy}
\end{itemize}

% Následující obrázek vygenerován přes PlantUML
%@startmindmap
%<style>
%mindmapDiagram {
%    node {
%      BackgroundColor white
%      LineColor black
%    }
%
%    arrow {
%      LineColor black
%    }
%}
%</style>
%* Obchodní korporace
%** Obchodní společnosti
%*** Osobní
%**** v.o.s
%**** k.s.
%*** Kapitálové
%**** s.r.o.
%**** a.s
%*** Evropská (akciová) společnost
%*** Evropské hospodářské zájmové sdružení (EHZS)
%** Družstva
%*** družstvo
%*** Evropská družstevní společnost
%@endmindmap
\begin{figure}[h]
\includegraphics[width=\textwidth]{obchodni-korporace.png}
\caption{Dělení obchodních korporací}
\end{figure}

\subsubsection{Ústav}
Jedná se o právnickou osobu založenou za účelem provozování společenské nebo hospodářské činnosti,
jejíž výsledky jsou dostupné všem rovnocenné na základě předem stanovených podmínek.
Zakládá se zakládací listinou nebo pořízením pro případ smrti, vzniká zápisem do veřejného rejstříku.
Statutárním orgánem je ředitel, kterého volí a odvolává správní rada. Nenaplňuje-li ústav svůj účel,
může jej soud zrušit na návrh osoby, která má na tom právní zájem.

\section{Organizace soudní moci v ČR \\ \small{postavení ústavního soudu,
základy procesního práva, \\ druhy civilního procesu, zásady soudních řízení}} 

\subsection{Soudnictví v ČR}
Kontrolní systém, který po porušení práva opět obnoví právní stav, který zabrání porušení práva.
Jeho základním orgánem je soud, jehož činnost je specifická svou nezávislostí, vázaností pouze 
zákonem a veřejností. Základním pramenem soudnictví v ČR je Ústava.

\subsection{Organizace moci soudní v ČR}

\subsubsection{Okresní soudy}
\begin{itemize}
    \item nejnižší článek soudní soustavy
    \item v prvním stupni rozhodují o věcech vyplývající z občanskoprávních, pracovních,
    rodinných nebo obchodních vztahů
    \item rozhodují o vině a trestu
\end{itemize}

\subsubsection{Krajské soudy}
\begin{itemize}
    \item druhý stupeň soudní soustavy
    \item v prvním stupni rozhodují ve věcech právně nesnadných, společensky významných, tam 
    kde jim to svěřuje zákon (ochrana soudnosti, autorská práva)
    \item jsou subjektem ve správním soudnictví, rozhodují zde v prvním stupni
    \item ve druhém stupni rozhodují, pokud k řízení v prvním stupni byl příslušný okresní soud
    \item tvořeny předsedou soudu, místopředsedy, soudci a přísedícími
    \item rozhodují samosoudci nebo v senátu
\end{itemize}

\subsubsection{Vrchní soudy}
\begin{itemize}
    \item Praha a Olomouc
    \item Působí jako soudy druhého stupně, pokud v prvním stupni rozhodoval soud krajský
    \item rozhodují v senátech tvořených předsedou a dvěma soudci
\end{itemize}

\subsubsection{Nejvyšší soud}
\begin{itemize}
    \item nejvyšší stupeň soudní soustavy
    \item sleduje a vyhodnocuje rozhodnutí soudů a zaujímá k nim stanoviska
    \item rozhoduje v prvním stupni, pokud tak stanoví zákon, je příslušný k řízení o dovolání
    \item rozhoduje v senátu, který tvoří předseda a dva soudci, popřípadě s počtem soudců stanoveným 
    zákonem
\end{itemize}

\subsection{Speciální soudy}

\subsubsection{Nejvyšší správní soud}
\begin{itemize}
    \item soudním orgánem specializovaným výlučně pro oblasti správního soudnictví
    \item rozhoduje o kasačních stížnostech směřujících proti pravomocným rozhodnutím krajských
    soudů ve správním soudnictví, rozhoduje ve věcech volebních, ve věcech rozpuštění politických
    stran a politických hnutí
    \item rozhoduje v senátu, který tvoří předseda a dva soudci
\end{itemize}

\subsubsection{Ústavní soud}
\begin{itemize}
    \item soudní orgán ochrany ústavnosti
    \item má 15 soudců, jmenováni prezidentem se souhlasem Senátu na 10 let
    \item odvolání není přípustné
    \item sídlí v Brně
    \item rozhoduje zejména o
    \begin{itemize}
        \item zrušení zákonů a jiných právních předpisů, jsou-li v rozporu s ústavním pořádkem
        \item o ústavní žalobě Senátu proti prezidentu republiky 
    \end{itemize}
\end{itemize}

\subsection{Soudci, přísedící a soudní čekatelé}

\subsubsection{Soudci}
\begin{itemize}
    \item může jím být každý občan ČR, způsobilý k právním úkonům, bezúhonný, kterému je alespoň 30 let
    \item musí mít vysokoškolské právnické vzdělání a odbornou justiční zkoušku
    \item jsou jmenování prezidentem doživotně
\end{itemize}

\subsubsection{Přísedící}
\begin{itemize}
    \item může jím být zvolen občan ČR se stejnými požadavky jako soudce 
    \item na okresní soudy volí zastupitelstvo obce, na krajské soudy zastupitelstvo kraje
    \item nemusí být vysokoškolské právnické vzdělání a odbornou justiční zkoušku
\end{itemize}

\subsubsection{Justiční čekatelé}
\begin{itemize}
    \item budoucí soudci, kteří po absolování magisterského studia prochází přípravnou službou,
    která trvá 3 roky
    \item po skončení se podrobí odborné justiční zkoušce
\end{itemize}

\subsection{Základy procesního práva}
\textbf{Procesní právo} = je souhrn právních norem, které regulují proces, postup, jak se domoci svých
subjektivních práv (hmotných práv) či jak realizovat své subjektivní povinnosti. Různé druhy procuse aplikace.
\begin{itemize}
    \item Procesní právo civilní (civilní řízení)
    \item Procesní právo správní (správní řízení)
    \item Procesní právo trestní (trestní řízení)
\end{itemize}

\subsection{Druhy civilního procesu, zásady soudních}

Základním pramenem právní úpravy civilního soudního procesu je Občanský soudní řád, který
upravuje postup soudu a účastníků v občanském soudním řízení a právní vztahy, které v důsledku
toho vznikají.

Dalším pramenem je 4. hlava Ústavy ČR + 5. hlava Listiny základních práv a svobod, Evropská úmluva
o ochraně lidských práv a základních svobod, nařízení ES.

\subsubsection{Druhy civiliního procesu}
\begin{itemize}
    \item \textbf{Řízení nalézací} - soud zjišťuje,
    co je právem a co je povinností a vydává o tom autoritativní rozhodnutí,
    které lze vykonat pomocí řízení vykonávacího
    \begin{itemize}
        \item \textbf{sporné} - spor mezi účastníky o existenci či neexistenci vzájemných práv
        a povinností (žaloby o náhradu škody)
        \item \textbf{nesporné} - cílem je úprava určitých právních vztahů ve veřejném zájmu
        (řízení ve věcech péče o nezletilé dítě, řízení opatrovnické, řízení o prohlášení za
        mrtvého)
    \end{itemize}
    
    \item \textbf{řízení exekuční} - dochází k realizaci toho, co bylo jako právo v konkrétním
    případě v nalézacím řízení soudem stanoveno
    
    \item \textbf{řízení zajišťovací}
    \item \textbf{řízení insolvenční}
    \item \textbf{řízení rozhodčí} - upravuje rozhodování majetkových sporů
\end{itemize}

\subsubsection{Zásady soudních řízení}
\begin{itemize}
    \item \textbf{zásada rovnosti účastníků} - strany mají před soudem rovná práva
    \item \textbf{zásada nezávislosti a nestrannosti}
    \item \textbf{zásada dispoziční} - procesní iniciativa je dána do rukou účastníků, kteří během
    soudního řízení činí procesní úkony
    \item \textbf{zásada oficiality} - výjimečně v řízení nesporném, průběh celého řízení určuje a řídí výhradně soud
    \item \textbf{zásada projednací} - povinnost účastníků poskytnout soudu potřebné údaje
    \item \textbf{zásada vyhledávací (vyšetřovací)} - pouze v řízení nesporné, soud je povinen 
    učinit vlastní vhodná opatření, která mu usnadní zjištění skutkového stavu v dané věci, a
    vyhledat k tomu potřebné důkazy
    \item \textbf{zásada volného hodnocení důkazů} - soud stanoví skutkový stav na základě vlastního
    hodnocení důkazů, jež není omezeno žádnými právními předpisy
    \item \textbf{zásada veřejnosti, ústnosti a přímosti} - každý má přístup k jednáním soudu, 
    podkladem pro rozhodnutí soudu je pouze to, co vyšlo najevo v průběhu ústního jednání a co se před nalézacím soudem přímo odehrálo
    \item \textbf{zásada jednotnosti řízení}
    \item \textbf{zásada rychlosti a hospodárnosti} - důraz na efektivnost jednání,
    aby byla věc co nejrychleji projednána a rozhodnuta
    \item \textbf{zásada formální (procesní) pravdy a materiální pravdy}
    \begin{itemize}
        \item \textbf{formální} - soud rozhoduje v konkrétním řízení podle tvrzení, skutečnosti 
        a důkazů, které mu poskytli účastníci řízení nebo v řízení vyšly najevo 
        \item \textbf{materiální} - procesní odpovědnost za zjištění skutečné pravdy nese soud.
        Soud má povinnost dobra se pravdy a musí sám pravdu vyšetřit
    \end{itemize}
\end{itemize}

\newpage

\section{Právní odpovědnost \\ \small{Pojem, předpoklady vzniku, funkce, druhy, právní odpovědnosti.}}

Právní odpovědnost jsou nepříznivé právní důsledky stanovené právní normou, které vznikají v důsledku
protiprávního jednání nebo stavu za právem stanovených podmínek. 
Jsou jedním z nástrojů dodržování zákonnosti. Rozsah a meze právní odpovědnosti stanovují zákony
v souladu s ústavním pořádkem a Listinou základních práv a svobod. Odpovědným subjektem je ten,
komu nepříznivé právní následky vznikají.

\begin{itemize}
    \item \textbf{retrospektivní odpovědnost} - odpovědnost za něco, co se už stalo (právní, trestní),
    odpovědnost za porušení povinnosti
    \item \textbf{aktivní (prospektivní) odpovědnost} - odpovědnost za splnění povinnosti (za budoucí jednání)
\end{itemize}

\subsection{Předpoklady vzniku právní odpovědnosti}

Právní odpovědnost vzniká jen tehdy, jsou-li splněny právními normami stanovené předpoklady
\begin{itemize}
    \item \textbf{objektivní}
    \begin{itemize}
        \item protiprávní jednání nebo protiprávní stav
        \item existence negativního důsledku protiprávního jednání (újma)
        \item příčinná souvislost mezi protiprávním jednáním a újmou (tzv. kauzální nexus)
    \end{itemize}
    
    \item \textbf{subjektivní}
    \begin{itemize}
        \item všechny předpoklady objektiví odpovědnosti
        \item zavinění - \textbf{úmysl} (přímý, nepřímý), \textbf{nedbalost} (vědomá, nevědomá)
    \end{itemize}
\end{itemize}

\subsection{Funkce právní odpovědnosti}
\begin{itemize}
    \item \textbf{preventivní} - předcházení porušení práva
    \item \textbf{reparační} - odstranění újmy tomu, komu byla protiprávním jednáním nebo stavem 
    způsobena
    \begin{itemize}
        \item sankční povinnost - náhrada vzniklé újmy, kompenzace (např. náhrada škody, 
        odstranění vad plnění)
        \item satisfakční povinnos - poskytnutí zadostiučinění (veřejná omluva)
    \end{itemize}
    \item \textbf{represivní} - postih odpovědného subjektu (represe), výchovný aspekt
\end{itemize}

\subsection{Druhy právní odpovědnosti}
\begin{itemize}
    \item podle charakteru společenských vztahů, k jejichž opravě slouží 
    (trestní, správní, občanskoprávní, pracovněprávní, ...)
    \item podle předpokladů jejího vzniku
    \begin{itemize}
        \item \textbf{subjektivní odpovědnost} - musí být splněny všechny předpoklady objektivního předpokladu
        a nic navíc. Např. trestní zavinění
        \item \textbf{objektivní odpovědnost} - k jejímu vzniku stačí objektivní předpoklady (např. 
        odpovědnost za vady a prodlení). Liberační důvody (např. vyšší moc) umožňují zprostit se
        odpovědnosti
    \end{itemize}
    
    \item podle toho, kdo nese nepříznivé právní následky
    \begin{itemize}
        \item \textbf{vlastní} - následky nese ten, kdo porušil právní povinnost
        \item \textbf{cizí} - následky nese někdo jiný než ten, kdo se provinil, 
        nebo je nesou oba dohromady (osoby nezletilé)
    \end{itemize}
    
    \item závazková (např. za vady a prodlení) a mimozávazková (např. trestní) 
    \begin{itemize}
        \item   podle toho, zda povinnost (tzv. primární povinnost), která byla protiprávním
        jednáním porušena, byla součástí závazkového vztahu
    \end{itemize}
    
    \item podle způsobu realizace
    \begin{itemize}
        \item k realizaci je vyžadován akt aplikace práva - realizuje se na základě 
        autoritativního zjištění
        \item k realizaci se akt aplikace práva nevyžaduje - realizuje se dobrovolně
    \end{itemize}
\end{itemize}

\subsection{Základy trestní odpovědnosti}
Trestní odpovědnost je odpovědnost za trestné činy. Jejich spáchání je právní skutečností 
zakládájící odpovědnostní vztah (trestněprávní vztah) mezi subjektem trestní odpovědnosti 
(pachatelem) a státem. Podstatou je povinnost podrobit se trestu.

\subsubsection{Skutková podstata trestného činu}
\begin{itemize}
    \item \textbf{objekt} - společenské zájmy, kterých se trestný čin dotýká, poškozuje je nebo je 
    ohrožuje
    \item \textbf{objektivní schránka} - skládá se z protiprávního jednání pachatele, následku
    a příčinného vztahu mezi dvěma stranami
    \item \textbf{subjekt} - fyzická osoba starší 15 let, která je v době činu příčetná
    \item \textbf{subjektivní schránka} - míra zavinění
\end{itemize}

\subsubsection{Trestní součinnost}
\begin{itemize}
    \item spolupachatelství
    \item účastenství - organizace, návod, pomoc
\end{itemize}

\subsubsection{Tresty}

Nepodmíněný trest odnětí svobody, podmíněný trest odnětí svobody, peněžitý trest,
prospěšné práce, zákaz činnosti, propadnutí věci, propadnutí majetku, domácí vězení

\subsubsection{Trestné činy}

Protiprávní činy, které trestní zákon označuje za trestné a které vykazují znaky uvedené v 
takovém zákoně.

\begin{itemize}
    \item \textbf{přečiny} - nedbalostní trestné činy, úmyslné trestné činy, za které hrozí
    odnětí svobody maximálně do výše 5 let
    \item \textbf{zločiny} - všechny trestné činy, které nejsou přečiny
\end{itemize}

\subsubsection{Protiprávní jednání (delikt)}

\begin{itemize}
    \item \textbf{trestný čin} - trest je uložený soudem
    \item \textbf{správní delikt} - uložena sankce správním orgánem (např. napomenutí,
    pokuta zákaz činnosti, ...). Lze je rozdělit na přestupky a jiné správní delikty
    \item \textbf{disciplinární delikt} - disciplinární sankce
    \item \textbf{soukromoprávní delikt} - povinnost nahradit újmu. Odpovědnostní povinnost má 
    tedy především reparační funkci
\end{itemize}

\subsubsection{Újma}

Majetková (škoda) nebo nemajetková. Negativní důsledek protiprávního jednání nebo stavu

\newpage

\section{Věci v právním smyslu}

Věc v právním slova smyslu je vše co je rozdílné od osoby a slouží k potřebě lidí. Plodem je to,
co věc pravidelně poskytuje ze své přirozené povahy, jak je dáno jejím obvyklým účelovým určením 
a přiměřeně k němu, ať s přičiněním člověka nebo bez něho.

Užitky jsou to, co věc pravidelně poskytuje ze své právní povahy. Lidské tělo ani jeho části,
třebaže byly od těla odděleny, nejsou věcí.

Živé zvíře má zvláštní význam a hodnotu již jako smysly nadaný tvor - živé zvíře není věcí a 
ustanovení o věcech se na živé zvíře použijí obdobně jen v rozsahu, ve kterém to neodporuje
jeho povaze.

Souhrn všeho, co osobně patří, se nazývá její \textbf{majetek}. Souhrn majetku a dluhů osoby
tvoří dohromady její \textbf{jmění}.

\subsection{Dělení věcí}

\subsubsection{Hmotné a nehmotné věci}
\begin{itemize}
    \item \textbf{hmotná věc} je ovladatelná část vnějšího světa, která má povahu samostatného
    předmětu
    \item \textbf{nehmotné věci} jsou práva, jejichž povaha to připouští, a jiné věci bez hmotné
    podstaty
    
    \item na ovladatelné přírodní síly, se kterými se obchoduje, se použijí přiměřená ustanovení o 
    věcech hmotných
\end{itemize}

\subsubsection{Movité a nemovité věci}
\begin{itemize}
    \item \textbf{nemovité věci} jsou pozemky a podzemní stavby se samostatným účelovým určením,
    jakož i věcná práva k nim a práva, která za nemovité věci prohlásí zákon
    \item stanoví-li jiný právní předpis, že určitá věc není součástí pozemku, a nelze-li takovou
    věc přenést z místa na místo bez porušení její podstaty, je i tato věc nemovitá.
    \item veškeré další věci, ať je jejich podstata hmotná nebo nehmotná, jsou brány jako 
    \textbf{věci movité}
\end{itemize}


\subsubsection{Další dělení věcí}
\begin{itemize}
    \item zastupitelné a nezastupitelné věci
    \item zuživatelné a nezuživatelné věci
    \item jednotlivé a hromadné věci
    \item dělitelné a nedělitelné věci
    \item genericky a individuálně určené věci
\end{itemize}

\newpage

\section{
    Věcná práva k věci vlastní, vlastnictví, držba \\ 
    \small{vznik, obsah, ochrana, omezení, spoluvlastnictví, SJM}
}

Vlastnické právo je soubor základních oprávnění vlastníka věc držet, užívat jí a používat její plody
a užitky, nakládat s ní - tzv. Vlastnická triáda (obsah VP)

\begin{itemize}
    \item oprávnění vlastníka (subjektivní právo vlastnické) - vlastnická triáda
    \item povinnosti vlastníka - nezneužívat vlastnické právo na úkor jiných, povinnost strpět
    omezení stanovená zákonem
\end{itemize}

\subsection{Věcné právo}
Právo k věci, které má \textbf{absolutní charakter}, tj. působí vůči všem a které přechází spolu
s ní na každého nového nabyvatele věci. Věcí se myslí věc v právním slova smyslu. Pro věcná práva
je typická jejich veřejnost (jejich existence se dá často ověřit ve veřejných seznamech, např.
katastr nemovitostí)

\subsubsection{Dělení věcných práv}
\begin{itemize}
    \item věcná práva k věcem vlastním
    \begin{itemize}
        \item vlastnictví
        \item spoluvlastnictví
        \item držba
    \end{itemize}

    \item věcná práva k věcem cizím - obecná práva, jejichž obsahem jsou pouze určitá oprávnění,
    částečné právní panství nad věcí, která oprávněnému nepatří
    \begin{itemize}
        \item právo stavby
        \item věcné břemeno
        \item zástavní právo
        \item zadržovací právo
        \item podzástavní právo
        \item správa cizího majetku
    \end{itemize}
\end{itemize}

\subsubsection{Nabytí vlastnického práva}
\begin{itemize}
    \item \textbf{Originální (původní)} - věc neměla vlastníka, a nebo měla, ale přesto vzniklo právo
    jako nové
    \begin{itemize}
        \item \textbf{vyrobením, zhotovením} - u věci co dříve neexistovala
        \item \textbf{vydržením} - uplynutím vydržecí lhůty (3 roky věci movité, 10 let věci nemovité)
        \item \textbf{přírůstkem} - plodonosné věci
        \item \textbf{zpracováním} - zpracovatel se v dobré víře domníval, že věc může zpracovat. 
        Vlastnické právo nabude ten, kdo má na nové věci větší podíl
        \item \textbf{smísením} - smísení vlastní věci s cizí, podobné jako zpracování
        \item \textbf{uplynutím doby v případě ztráty věci} - pokud není znám vlastník, nálezce
        povinen odevzdat obecnímu úřadu (nálezné 10\%), pokud se vlastník do roka nepřihlásí,
        vlastníkem se stává stát
        \item \textbf{opuštěním věci} - vlastnické právo zaniká projevem vůle vlastníka,
        vlastníkem se stává stát
        \item \textbf{nalezením věci skryté} - vlastníkem se stává stát, nálezce je povinen odevzdat
    \end{itemize}
    
    \item \textbf{Derivativní (odvozené)}
    \begin{itemize}
        \item \textbf{převod} - projev vůle, vlastník převede vlastnické právo (právní jednání)
        \begin{itemize}
            \item věci movité - převádí se tradicí
            \item věci nemovité - vlastnické právo se nabývá vkladem do katastru nemovitostí
        \end{itemize}
        
        \item \textbf{přechod vlastnického práva} - např. při smrti - dědictví (právní událost)
    \end{itemize}
\end{itemize}

\subsubsection{Ochrana vlastnického práva}
\begin{itemize}
    \item \textbf{poskytovaná soudem}
    \begin{itemize}
        \item žalobou na vydání věci - movité - vydání, nemovité - vyklizení
        \item žalobou popírací - míří proti zásahům do vlastnického práva, které vlastníka
        ruší nebo omezují při výkonu jeho práv
    \end{itemize}
    
    \item \textbf{poskytovaná orgánem státní správy}
    \item \textbf{svépomocná ochrana} - možnost přímo chránit svá práva. Musí být splněny podmínky
    (svépomoc přiměřená zásahu, bezprostřední hrozba zásahu, neoprávněnost zásahu)
\end{itemize}

\subsubsection{Omezení vlastnického práva - pouze zákonem}

\begin{itemize}
    \item \textbf{úprava sousedských práv} - zákaz imisí (tj. např. zákaz obtěžování sousedů hlukem,
    zápachem, kouřem, ...)
    \item \textbf{možnost použít věc bez vlastníkova svolení} - pouze při stavu nouze a z naléhavého
    veřejného zájmu (na dobu nezbytně nutnou, v nezbytné míře a za náhradu)
    \item \textbf{vyvlastnění} - změna VP na základě vydání individuálního právního aktu,
    je možné pouze ve veřejném zájmu, na základě zákona, za náhradu a nelze-li dosáhnout účelu jinak
    (např. dohodou)
\end{itemize}


\subsubsection{Zánik vlastnického práva}

\begin{itemize}
    \item \textbf{absolutně} (nenabývá ho někdo jiný) nebo \textbf{relativně} (nabývá ho někdo jiný)
    \item \textbf{projevem vůle vlastníka} - smlouvou, opuštěním věci, zničením (spotřebováním) věci
    \item \textbf{nezávisle na vůli vlastníka} - smrtí vlastníka, vydržením, vyvlastněním...
\end{itemize}

\subsubsection{Držba}

Faktický stav, při kterém držitel nakládá s věcí jako s věcí vlastní. Držitel je přitom osoba, která
věc či právo drží a je od vlastníka odlišná.

NOZ stanoví vyvratitelnou právní domněnku, že držba je řádná, poctivá a pravá

\begin{itemize}
    \item \textbf{řádná} - platný právní důvod
    \item \textbf{poctivá} - osoba je v dobré víře, že jí věc patří
    \item \textbf{pravá} - nesmí být potajmu nebo lstí
\end{itemize}

\subsubsection{Vydržení}

Poctivý držitel, který právo vykonává po stanovenou dobu, v oprávněném domnění, že mu náleží, je
může vydržet. K vydržení se vyžaduje pravost držby a aby se držba zakládala na platném právním
původu.

Podmínkou pro vydržení vlastnického práva je nepřerušená držba trvající 3 roky,
u nemovitostí pak 10 let.

Pokud osoba neprokáže právní důvod své držby, pak se jedná o mimořádné vydržení, pro které platí
dvojnásobná vydržecí doba.

\subsection{Spoluvlastnictví, společné jmění manželů}

\subsubsection{Reálné spoluvlastnictví}
Spoluvlastnictví mají ve svém vlastnictví určenou část věci.

\subsubsection{Podílové spoluvlastnictví}
Věc není rozdělená, dělí se pouze vlastnické právo.

\subsubsection{Vznik spoluvlastnictví}
\begin{itemize}
    \item vzniká stejným způsobem jako vlastnické právo, jeho předmětem je ale společná věc
    \item míru práv a povinností spoluvlastníků určují jejich podíly (dané procentem)
    \item pokud výše podílů není dohodnuta nebo stanovena právním předpisem, podíly jsou stejné
\end{itemize}

\subsubsection{Obsah spoluvlastnictví}
Práva a povinnosti se rozdělují do 3 skupin, podle toho, čeho se týkají
\begin{itemize}
    \item \textbf{vzájemný vztah spoluvlastníků} - rozhodující není dohoda všech, ale jen většiny
    podle podílů (tzv. princip majorizace)
    \item \textbf{vztah všech spoluvlastníků ke třetím osobám ohledně společné věci} - spoluvlastníci
    jsou jediný subjekt, \textbf{jednají společně a nerozdílně}
    \item \textbf{Vztah jednoho spoluvlastníka ke třetím osobám ohledně jeho podílu} - převod podílu
    na osoby blízké nevyžaduje souhlas ostatních spoluvlastníků, převod na osoby jiné ho vyžaduje
\end{itemize}

\subsubsection{Zrušení spoluvlastnictví}
\begin{itemize}
    \item \textbf{dohodou spoluvlastníků} - musí se týkat vždy celého předmětu spoluvlastnictví
    \item \textbf{rozhodnutím soudu} - na návrh některého spoluvlastníka, nedojde-li k dohodě,
    soud respektuje způsoby vypořádání stanovené v občanském zákoníku
    \begin{itemize}
        \item oddělení ze spoluvlastnictví, podle výše podílu (pokud je to možné)
        \item převedením vlastnického práva jednomu nebo více spoluvlastníkům vyplacením ostatních
        \item prodej věci a rozdělení výtěžku
    \end{itemize}
    
    Občanský zákoník rozlišuje spoluvlastnictví podílové, bytové, přídatné a společenství jmění
\end{itemize}

\subsection{Společné jmění manželů}
Společné jmění manželů je součástí společenství jmění. Základní úprava majetkových vztahů 
mezi manželi. Od podílového spoluvlastnictví se liší tím, že podíly nejsou určeny v době 
jeho trvání, ale k jejich určení dochází až při zániku společného jmění, tj. představuje 
ideální formu vlastnictví bez vymezení podílu. Společné jmění může vzniknout pouze mezi manželi.

\subsubsection{Vznik společného jmění manželů}
\begin{itemize}
    \item vzniká okamžikem uzavření manželství
    \item dohodou manželů (forma notářského zápisu) může být odložen až ke dni zániku manželství
\end{itemize}

\subsubsection{Předmět a rozsah společného jmění manželů}
\begin{itemize}
    \item \textbf{majetek} - to, čeho nabyl jeden z manželů, nebo čeho nabili oba manželé společně
    za trvání manželství. Jedná se např. o věci, majetková práva, pohledávky, podíl manžela v 
    obchodní společnosti nebo družstvu...
    \item \textbf{dluhy} - převzaté za trvání manželství
    \item do společného jmění manželů nepatří majetek, nabitý dědictvím nebo darem a věci, které
    slouží k osobní spotřebě jen jednoho z manželů. Nabytí podle právních předpisů o restituci 
    majetku
    \item předmět a rozsah společného jmění manželů může vyplývat
    \begin{itemize}
        \item ze zákonného režimu
        \item ze smluvního režimu
        \item z režimu založeného na základě rozhodnutí soudu
    \end{itemize}
\end{itemize}

\subsubsection{obsah společného jmění manželů}
Práva a povinnosti náleží oběma manželům společně a nerozdílně = solidarita práv a povinností,
také solidarita právních jednání

\begin{itemize}
    \item \textbf{V jejich vzájemném vztahu} - manželé mají stejné právo společně užívat majetek 
    společného jmění
    \item \textbf{Ve vztahu k třetím osobám} - běžné záležitosti týkající se společného majetku může 
    vyřizovat každý z manželů, v ostatních záležitostech je třeba souhlasu obou manželů
\end{itemize}

Společné jmění manželů může být ve třech režimech:

\begin{itemize}
    \item \textbf{zákonný} - vše, co nabili za dobu trvání manželství s výjimkou věcí osobní potřeby,
    darů, dědictví, náhrady nemajetkové újmy na svých přirozených právech, co nabyli ze svého 
    výlučného jmění a zisk ze všeho předešlého. 
    Součástí jsou i dluhy převzaté za trvání manželství
    
    \item \textbf{smluvená} - smlouva o manželském majetkovém režimu formou veřejné listiny, ne
    úpravy správy majetku. Režim funguje buď v režimu oddělených jmění, v režimu vyhrazujícím vznik
    SJM ke dni zániku manželství, nebo v režimu rozšíření nebo zúžení rozsahu SJM v zákonném režimu.
    
    \item \textbf{režim založený a rozhodnutí soudu} - nastává pouze ze závažných důvodů:
    \begin{itemize}
        \item manželův věřitel požaduje zajištění své pohledávky v rozsahu přesahujícím hodnotu toho,
        co náleží výhrazně tomuto manželi
        \item manžela lze považovat za marnotratného
        \item manžel soustavně nebo opakovaně podstupuje nepřiměřená rizika
    \end{itemize}
\end{itemize}

SJM zaniká zánikem manželství, nebo za trvání manželství, když je prohlášen konkurz na majetek
jednoho z manželů, nebo jednomu z manželů je uložen trest propadnutí majetku.

\newpage

\section{
    Věcná práva k věci cizí, právo stavby, \\  správa cizího majetku \\
    \small{reálná břemena a služebnosti, právo zadržovací a zástavní}
}

\subsection{Věcná práva k věcem cizím}
Věcná práva, jejichž obsahem jsou pouze určitá oprávnění, částečné právní panství nad věcí, která
oprávněnému nepatří

\begin{itemize}
    \item právo stavby
    \item věcné břemeno
    \item zástavní právo
    \item zadržovací právo
    \item podzástavní právo
    \item správa cizího majetku
\end{itemize}

\subsubsection{Věcná břemena (služebnosti)}
Jejich podstatou je omezení vlastníka věci ve prospěch někoho jiného tak, že je povinen něco strpět,
něčeho se zdržet, něco konat, nebo něco dát. Věcná břemena slouží k tomu, aby oprávněný mohl využít
určitou část užitné hodnoty cizí věci, pro vlastníka to znamená, že je naopak povinen něco dát, konat,
strpět, nebo se něčeho zdržet. Věcná břemena spočívají v dlouhodobém nebo opakovaném právním jednání.

\begin{itemize}
    \item \textbf{služebnosti} - vlastník věci je povinen něco strpět, nebo se něčeho zdržet 
    (pasivní činnost). Např. neoplotit část pozemku.
    \item \textbf{reálná břemena} - vlastník věci má povinnost něco dát nebo konat (aktivní činnost)
\end{itemize}

\paragraph{Služebnosti}

Oprávněný ze služebnosti se může domáhat ochrany svého práva. Oprávněný se podílí na nákladech.
Služebnost vázne na věci i v případě, že změní vlastníka.
Nabytí služebnosti je možné:

\begin{itemize}
    \item smlouvou
    \item pořízením pro případ smrti, tj. závěť nebo dědická smlouva
    \item vydržením
    \item ze zákona
    \item rozhodnutím orgánu veřejné moci (např. soudní rozhodnutí)
\end{itemize}

\textbf{Služebnost vzniká:}

\begin{itemize}
    \item zápisem do seznamu - pokud jsou věci zapsané ve veřejném seznamu
    \item účinností smlouvy - není-li věc  zapsaná ve veřejném seznamu
\end{itemize}

\textbf{Druhy služebnosti:}

\begin{itemize}
    \item \textbf{pozemkové služebnosti} - zřizují se ve prospěch určitého panujícího pozemku
    \begin{itemize}
        \item služebnost inženýrské sítě
        \item opora cizí stavby
        \item právo na svod dešťové vody
        \item služebnost stezky, průhonu a cesty
        \item právo pastvy
    \end{itemize}
    
    \item \textbf{osobní služebnosti} - zřizují se ve prospěch určité osoby
    \begin{itemize}
        \item užívací právo - právo užívat cizí věc
        \item požívací právo - právo užívat cizí věc a brát z ní plody a užitky
        \item služebnost bytu
    \end{itemize}
\end{itemize}

\textbf{Zánik služebnosti:}
\begin{itemize}
    \item \textbf{trvalá změn} - služebná věc již nemůže sloužit panujícímu pozemku nebo
    oprávněné osobě
    \item \textbf{smlouvou} - dohodnou-li se strany
    \item smrtí oprávněného - osobní služebnost
    \item uplynutím doby, na kterou byla sjednána
    \item rozhodnutím orgánu veřejné moci - ze zákona, zánikem věci
\end{itemize}

\paragraph{Reálná břemena}

\begin{itemize}
    \item vlastník věci je zavázán vůči oprávněné osobně něco dávat nebo konat
    \item lze zřídit pouze k věci zapsané do veřejného seznamu (typicky tak věci nemovité)
    \item časově neomezené - musí být vykupitelné, jinak časově omezené
    \item vznik zápisem do veřejného seznamu, zánik obdobně jako u služebností
    \item např. poskytovat část plodin, které se urodili na pozemku vlastníka
\end{itemize}

\subsection{Správa cizího majetku}

\begin{itemize}
    \item každý, komu je svěřena správa majetku, který mu nepatří, ve prospěch někoho jiného 
    (beneficienta), je \textbf{správcem} cizího majetku
    \item správce právně jedná jako zástupce vlastníka, činnost správce je zásadně úplatná
    \begin{itemize}
        \item \textbf{správa cizího majetku prostá} - je povinností správce pečovat o zachování
        podstaty a účelu majetku
        \item \textbf{správa cizího majetku plná} - správci vznikají širší oprávnění - správce může
        se spravovaným majetkem učinit vše, co je potřebné k jeho zachování, zhodnocení nebo 
        rozmnožení
    \end{itemize}
\end{itemize}

\subsubsection{Povinnosti správce vůči beneficientovi}
\begin{item}
    \item povinnost správce jednat s péčí řádného hospodáře - čestně, věrně, prozíravě a pečlivě
    se zřetelem k účelu, jehož má být dosaženo
    \item správce předkládá 1x ročně vyúčtování
\end{item}

\subsubsection{Skončení správy}
\begin{itemize}
    \item odstoupením, odvoláním, omezením svéprávnosti nebo usvědčením úpadku správce
    \item uplynutím doby, na kterou byla zřízena, dosažením účelu nebo zánikem práva
    beneficienta ke spravovanému majetku
    \item povinnost předložit vyúčtování při skončení
\end{itemize}

\subsection{Svěřenecký fond}
\begin{itemize}
    \item podstata je v tom, že jeho zakladatel vyčlení ze svého majetku určitou část a svěří jí
    svěřeneckému správci, který majetek spravuje ve prospěch určité osoby (\textbf{obmyšleného})
    či za konkrétním (např. charitativním) účelem
    \item vzniká smlouvou nebo pořízením pro případ smrti
    \item správce je oprávněn ke svěřenému majetku vykonávat práva, která přísluší vlastníkovi,
    ale nenabývá svěřený majketek do vlastnictví, tudíž nemůže být použit na jeho dluhy. Zástavní
    právo toto zabezpečuje
    \item zánik - uplynutím doby, dosažením účelu nebo rozhodnutím soudu
\end{itemize}

\subsection{Právo stavby}
\textbf{speciální věcné právo}

\begin{itemize}
    \item nezáleží na tom, zda se  jedná o stavu již zřízenou nebo dosud nezřízenou
    \item právo stavby je bráno jako \textbf{věc nemovitá}
    \item nelze zřídit k pozemku, na kterém vázne právo příčící se účelu stavby
\end{itemize}

\subsection{Právo zástavní}
\begin{itemize}
    \item rozlišujeme zástavního věřitele a zástavního dlužníka
    \item zástavní právo zabezpečuje věřiteli ochranu proti každému, kdo by jej ve výkonu tohoto
    práva ohrožoval
    \item má akcesorickou povahu - jeho existence je závislá na existenci hlavního závazku
    \item má dvě funkce: zajišťovací a uhrazovací
    \item zajišťovací - v době, kdy dluh není splacen
    \item uhrazovací - nastupuje poté, co dlužník nesplatí svůj dluh
    \item předmětem může být jakákoliv věc, jakákoliv hodnota, se kterou lze obchodovat
    \item vznik - smlouva o zřízení zástavního práva, určení zástavy, určení zajišťovacího dluhu,
    určení jistiny
\end{itemize}

\subsection{Právo zadržovací}
\begin{itemize}
    \item slouží k zajištění splatného dluhu
    \item funkce je pouze zajišťovací, nikoliv uhrazovací
    \item nevzniká na základě smlouvy
    \item zadržovatel má povinnost o věc pečovat
\end{itemize}

\newpage

\section{Podnikatel, podnikání, spotřebitel \\ \small{obchodní firma a sídlo podnikatele, závod}}

\subsection{Podnikání}

Výdělečná činnost vykonávaná soustavně, samostatně, na vlastní odpovědnost, vlastním jménem a za 
účelem dosažení zisku.

\subsection{Podnikatel}

Ten, kdo vykonává tuto činnost na základě živnostenského nebo jiného oprávnění a je zapsán v obchodním
rejstříku. Může jím být fyzická i právnická osoba. 

\subsubsection{Identifikační znaky}
\begin{itemize}
    \item obchodní firma / jméno příjmení
    \item sídlo / bydliště
    \item IČO
    \item údaj o zápisu v obchodním rejstříku
\end{itemize}

Základní zásadou je nezaměnitelnost a neklamnost těchto znaků.

\subsubsection{Spotřebitel}

Fyzická osoba, která mimo rámec svého podnikání či samostatného výkonu svého povolání uzavírá 
smlouvu s podnikatelem

\subsection{Obchodní firma}
\begin{itemize}
    \item jméno, pod kterým je podnikatel zapsán do obchodního rejstříku (nezapsaná osoba nemá firmu)
    \item u fyzických osob je to jméno, příjmení a dodatek odlišující osobu podnikatele či
    druh podnikání
    \item u právnické osoby je to název a dodatek označující právní formu
    \item je to pojmenování podnikatele, nejedná se o právní subjekt (podnik, závod)
    \item zásada jedné firmy: podnikatel smí mít pouze jednu firmu
    \item zásada staré firmy: pokud se jméno člověka-podnikatele změní, může používat v obchodní firmě
    své staré jméno bez potřeby připojit dodatek s novým jménem, změnu však musí uveřejnit
    \item obchodní firma může být převedena smlouvou (samostatně a volně), a nebo si podnikatel
    může firmu ponechat a poskytnout ji jinému podnikateli do užívací smlouvy (franšízing)
    \item \textbf{základní principy při vytváření firmy}
    \begin{itemize}
        \item \textbf{nesmí být zaměnitelná}
        \item \textbf{nesmí působit klamavě}
        \item \textbf{je stanovená priorita} (práva má ten, kdo ji použil jako první)
        \item \textbf{splňuje zásadu relativní výlučnosti} (může mít některé prvky shodné, ale 
        musí se alespoň v něčem i lišit)
    \end{itemize}
    \item \textbf{ochrana obchodní firmy} - každý, kdo byl dotčen na svých právech neoprávněným 
    užíváním firmy, se může proti neoprávněnému uživateli domáhat, aby odstranil závadný stav,
    popřípadě nahradil vzniklou újmu (hmotnou - škody, nebo nehmotnou - pověst)
\end{itemize}

Obchodní firma i jméno podnikatel zapsaného do OR mají \textbf{trojí význam}

\begin{itemize}
    \item \textbf{prvek identity} podnikatele
    \item \textbf{ochranné označení} - může být součástí jeho goodwillu
    \item speciální typ hmotného statku vyznačující \textbf{osobnost podnikatele} (zvláště pokud
    se shoduje jméno podnikatele s názvem firmy)
\end{itemize}

\subsubsection{Sídlo}
\begin{itemize}
    \item sídlo je významné pro příslušný soud, správce daně atd.
    \item každá právnická osoba musí mít sídlo určené adresou
    \item právní důvod užívání prostor, kde má právnická osoba - nájemní smlouva, vlastnictví nebo 
    prohlášení vlastníka nemovitosti
    \item u fyzických osob je sídlo určené adresou zapsanou v obchodním rejstříku či jiné evidenci
    \item pokud se nezapisuje osoba jako podnikatel, má sídlo jako adresu závodu, nebo bydliště
\end{itemize}

\subsubsection{Obchodní závod}
\begin{itemize}
    \item organizovaný soubor jmění, který podnikatel vytvořil a který z jeho vůle slouží 
    k provozování
    \item není to právní subjekt, který podniká (to je podnikatel), ale ten předmět, se kterým
    podnikatel podniká
    \item v závodu se propojují osobní, hmotné a nehmotné složky podnikání (např. goodwill - duše
    závodu)
    \item je to věc ve právním slova smyslu (věc hromadná)
    \item podnikateli je umožněna dispozice se závodem - smlouva o koupi, pachtu, ...
\end{itemize}

\newpage
\section{Jednání podnikatele a za podnikatele, zastoupení podnikatele, prokura}

\subsection{Jednání podnikatele - fyzické osoby}
\begin{itemize}
    \item \textbf{přímé (osobní) jednání dané fyzické osoby}
    \item \textbf{nepřímé jednání dané fyzické osoby - zastoupení}
    \begin{itemize}
        \item obecná úprava zastoupení - zákonný zástupce, opatrovník, člen domácnosti,
        zástupce stanovený smlouvou
        \item zvláštní zastoupení podnikatele fyzické osoby - \textbf{prokurista}, zmocněnec 
        podnikatele při provozu obchodního závodu, zastoupení v provozovně podnikatele, 
        zastoupení vedoucím odštěpeného závodu, likvidátor, insolvenční správce
    \end{itemize}
\end{itemize}

\subsection{Jednání podnikatele - právnické osoby}
\begin{itemize}
    \item právnické osoby zastupují její orgány, ti mohou být individuální nebo kolektivní 
    (rozhoduje většina)
    \item nejdůležitějším orgánem právnické osoby je \textbf{statutární orgán}, který může 
    právnické osoby zastupovat ve všech záležitostech a má vůči jiným orgánům zbytkovou
    působnost (náleží mu působnost kterou zákon, či jiné prostředky nestanovili jinému orgánu)
    \item jiným orgánem právnické osoby je například dozorčí rada nebo výbor pro audit, právnická
    osoba může vytvořit i další orgány nezapsané v obchodním rejstříku - jejich členové pravomoci
    obdobné zaměstnancům
\end{itemize}

\subsection{Přímé nebo nepřímé zastoupení podnikatele}
\begin{itemize}
    \item \textbf{přímé zastoupení} - zástupce jedná \textbf{jménem} a \textbf{na účet} zastoupeného
    \item \textbf{nepřímé zastoupení} - zástupce jedná \textbf{svým jménem}, ale na \textbf{na účet}
    zastoupeného
\end{itemize}

\subsection{Prokura}

Zvláštní případ smluvního zastoupení podnikatele (velmi široká ,,podnikatelská plná moc'')

\begin{itemize}
    \item podnikatel zmocňuje prokuristu ke všem právním jednáním, ke kterým dochází při provozu
    obchodního závodu (filiální prokura - k jednou závodu, pobočce)
    \item prokura se povinně zapisuje do obchodního rejstříku s deklaratorním účinkem
    \item smrtí podnikatele ani výmazem z obchodního rejstříku prokura nezaniká, zaniká ukončením
    zastupitelského smluvního vztahu zakládajícího prokuru (např. převod, pacht závodu)
    \item musí mít \textbf{písemnou formou} a je účinná jejím udělením
    \item prokuristou může být \textbf{pouze fyzická osoba} a uděluje ji ten, kdo je zapsaný do
    obchodního rejstříku
    \item podnikatel může \textbf{udělit prokuru několika osobám} - každý z nich jedná samostatně,
    ledaže se jedná o tzv. \textbf{kolektivní prokuru}
    \item \textbf{nelze ji omezit}, lze ji pouze rozšířit (na zcizování a zatěžování nemovitostí)
    \item prokura musí být vykonávána \textbf{osobně a s péčí řádného hostpodáře}
\end{itemize}

\subsection{Jednání \textit{ultra vires} - nad rámec, překročení rozsahu zástupčího oprávnění}

\begin{itemize}
    \item \textbf{zákonné zastoupení} - zastoupený musí překročení \textbf{schválit}, jinak je z 
    jednání zavázán zástupce
    \item \textbf{smluvní zastoupení} - zastoupený má povinnost \textbf{reakce} (aktivní vyslovení
    \textbf{ne}souhlasu), jinak je z jednání zavázán
    \item \textbf{podnikatelé} - překročení zástupčího oprávnění podnikatele zavazuje \textbf{vždy}
\end{itemize}

\newpage

\section{Živnostenské podnikání, neživnostenské podnikání}

\subsection{Živnost}
\begin{itemize}
    \item soustavná činnost, provozovaná samostatně, vlastním jménem, na vlastní odpovědnost,
    za účelem dosažení zisku a za podmínek stanovených živnostenským zákonem
\end{itemize}

\subsection{Podmínky provozování živnosti}
\begin{itemize}
    \item všeobecné
    \begin{itemize}
        \item plná svéprávnost - dosažení zletilosti (18 let nebo manželství)
        \item bezúhonnost - ten, kdo nebyl trestán v souvislosti s podnikáním
    \end{itemize}
    \item zvláštní podmínky - odborná nebo jiná způsobilost (v případech odborných živností)
\end{itemize}

\subsection{Překážky provozování živnosti}
\begin{itemize}
    \item \textbf{překážky soukromoprávní}
    \begin{itemize}
        \item pokud byl na majetek prohlášen konkurs
        \item pokud dojde k zamítnutí insolvenčního návrhu proto, že majetek dlužníka nebude
        postačovat k náhradě insolvenčního řízení
        \item pokud je vydáno rozhodnutí o zrušení konkursu proto, že majetek dlužníka je zcela 
        nepostačující pro uspokojení věřitelů
    \end{itemize}
    
    \item \textbf{překážky veřejnoprávní}
    \begin{itemize}
        \item pokud byl soudem uložen trest nebo zákaz
    \end{itemize}
\end{itemize}

\subsection{Provozování živnosti prostřednictvím odpovědného zástupce}
\begin{itemize}
    \item odpovědným zástupcem může být jen fyzická osoba, která splňuje všeobecné i zvláštní
    podmínky provozování živnosti, odpovídá za řádný provoz živnosti
    \item \textbf{obligatorní odpovědný zástupce} - zákon uloží podnikateli ho stanovit pokud:
    \begin{itemize}
        \item podnikatelem je fyzická osoba, která nesplňuje zvláštní podmínky provozování živnosti
        \item podnikatelem je právnická osoba se sídlem v ČR
        \item podnikatelem je zahraniční právnická osoba
    \end{itemize}
    \item \textbf{fakultativní odpovědný zástupce} - u osoby, která by sice mohla provozovat živnost
    osobně, ale z jakéhokoliv důvodu to nechce
\end{itemize}

\subsection{Povinnosti podnikatele při provozování živnosti}
\begin{itemize}
    \item zajistit účast odpovědného zástupce při provozování živnosti v potřebném rozsahu
    \item viditelně označit obchodní firmou nebo jinak objekt, v němž má sídlo
    \item prokázat kontrolnímu orgánu způsob nabytí prodávaného zboží atd.
    \item povinnost, aby na provozovně byla osoba hovořící česky, slovensky
    \item povinnost vydat doklady o prodeji zboží
\end{itemize}

\subsection{Dělení živností}
\begin{itemize}
    \item \textbf{ohlašovací} 
    \begin{itemize}
        \item řemeslné - nutná odborná způsobilost, vyučení v oboru (cukrářství, pekařství,
        kadeřnictví)
        \item vázané - nutná odborná způsobilost (účetnictví, provozování autoškoly, masáže)
        \item volné - žádná odborná způsobilost (pronájem, návrhářství, modeling)
    \end{itemize}
    \item \textbf{koncesované} - mohou být provozovány jen na základě státního povolení - koncese 
    (výroba alkoholických nápojů, prodej zbraní, výroba paliv)
    \item \textbf{co umožňuje živnostenské podnikání}: vyrábět, prodávat, prodávat výrobky i jiných
    výrobců, poskytování služeb
\end{itemize}

\subsection{Živnostenské oprávnění}
\begin{itemize}
    \item oprávnění provozovat živnost
    \item u ohlašovacích živností vzniká dnem ohlášení
    \item u právnických osob vzniká zápisem do obchodního rejstříku
    \item \textbf{prokazování oprávnění}: výpisem z obchodního rejstříku, rozhodnutím o udělení
    koncese
    \item \textbf{živnostenské úřady} vydávají a odebírají oprávnění
\end{itemize}

\subsection{Neživnostenské oprávnění}
\begin{itemize}
    \item \textbf{činnost, která není v živnostenském zákoně}, je upravena jinými předpisy než 
    zákonem o živnostenském podnikání. Může být vykonáváno FO, PO, FO + PO.
    \item \textbf{nepodnikatelské činnosti}
    \begin{itemize}
        \item \textbf{pronájem nemovitostí}, bytů a nebytových prostor není žívností v případě,
        že vedle pronájmu není poskytovaná i jiná činnost
        \item činnosti, které nejsou právně regulovány - služby k \textbf{uspokojování sexuálních
        potřeb}, \textbf{služby přírodních léčitelů}
        \item \textbf{využívání výsledků duševních tvůrčí činnosti chráněných zvláštními zákony, 
        jejich původci nebo autory}
        \begin{itemize}
            \item pokud tuto činnost vykonávají autoři, nepotřebují ani ničí souhlas, ani získání
            živnostenského či jiného podnikatelského oprávnění
            \item podmínkou je, že vydávají, rozmnožují či rozšiřují svá autorská díla na své
            náklady
            \item jiná osoba než autor potřebuje k takové činnosti vedle souhlasu autora i příslušné
            oprávnění
        \end{itemize}
        \item \textbf{restaurování kulturních památek} nebo jejich částí, které jsou díly výtvarných
        umění nebo uměleckořemeslnými pracemi
        \item \textbf{provádění archeologických výzkumů}
    \end{itemize}
    \item \textbf{podnikatelské činnosti} - musí mít podnikatelské oprávnění
    \begin{itemize}
        \item \textbf{neživnostenské podnikání fyzických osob} - svobodná povolání (lékaři, noráři, 
        auditoři, advokáti, právníci, veterináři atd. - nutné vysokoškolské vzdělání + praxe +
        atestace)
        \item \textbf{neživnostenské podnikání právnických osob} - činnost bank, výroba elektřiny,
        obchod s elektřinou, drážní doprav, televizní vysílání, lesnictví, výroba léčiv, činnosti 
        vyhrazené ze zákona státu (pošta)
    \end{itemize}
\end{itemize}

\newpage

\section{Veřejné rejstříky \\ \small{význam zápisu, obchodní rejstřík, sbírka listin}}

Informační systém veřejné správy, kam se zpisují zákonem stanovené údaje o osobách, o kterých to 
zákon stanoví (všechny právnické osoby a některé fyzické osoby)

\subsection{Veřejné rejstříky}

Veřejným rejstříkem se rozumí 6 taxativně vymezených rejstříků. Veřejné rejstříky jsou vedeny
rejstříkovým soudem v elektronické podobě

\begin{itemize}
    \item \textbf{spolkový rejstřík}
    \item \textbf{nadační rejstřík}
    \item \textbf{rejstřík ústavů}
    \item \textbf{rejstřík společenství vlastníků jednotek}
    \item \textbf{obchodní rejstřík}
    \item \textbf{rejstřík obecně prospěšných společností}
\end{itemize}

\subsubsection{Princip formální publicity}
\begin{itemize}
    \item veřejný rejstřík \textbf{je každému přístupný}, každý má právo do něj nahlížet, pořizovat 
    si z něj opisy a výpisy (výjimkou je rodné číslo,k které se nezveřejňuje, pokud není součástí
    listin)
\end{itemize}

\subsubsection{Princip materiální publicity} - zákon chrání každého, kdo důvěřuje údajům zapsaným
ve VR a obsahu listin ve Sbírce listin
\begin{itemize}
    \item \textbf{pozitivní} - skutečnosti, zapsané v obchodním rejstříku, jsou účinné vůči třetím
    osobám až ode dne zveřejnění
    \item \textbf{negativní} - nebyla-li některá skutečnost do VR zapsána, ačkoliv tam zapsána být
    měla, tak zapsané skutečnosti by v takovém případě byly účinné vůči třetím osobám, přestože
    by neodpovídaly skutečnosti
\end{itemize}

\subsubsection{Právní účinky zápisů ve veřejném rejstříků}
\begin{itemize}
    \item \textbf{konstitutivní zápis} - má za následek vznik, změnu nebo zánik určitého právního
    stavu. Aby daná skutečnost nastala, je nezbytné aby byla zapsána do VR. Bez zápisu by právní
    stav vůbec nenastal. (např. zápis vzniku či zániku obchodní společnosti)
    \item \textbf{deklaratorní zápis} - právní účinky nastávají již na základě právní skutečnosti
    před zapsáním do VR. Zápis do VR o tomto stavu pouze informuje. (např. změna člena orgánu PO)
\end{itemize}

\subsubsection{Rejstříkové řízení} (ve věcech VR, o zápisu do VR, patří mezi nesporná řízení)
\begin{itemize}
    \item \textbf{je vedeno rejstříkovými soudy - krajské soudy}
    \item řízení ve věcech veřejného rejstříku se zpravidla zahajuje na návrh. Má-li však být
    dosažena shoda mezi zápisem a skutečným stavem, lze řízení zahájit i bez návrhu
    \item \textbf{návrh na zápis musí být podán} prostřednictvím tzv. inteligentních formulářů -
    elektronické, podpis na formuláři musí být úředně ověřen (u elektronického návrhu musí obsahovat
    potřebné elektronický podpis), musí být podán oprávněnou osobou, musí být srozumitelný a
    obsahovat potřebné náležitosti. Poté jsou zaslány rejstříkovému soudu.
\end{itemize}

\subsubsection{Povinnost zapsané osoby vůči veřejnému rejstříku}
\begin{itemize}
    \item vedení aktuálních informací, podat návrh na zápis do 15 dnů ode dne, kdy došlo k nějaké změně. Pokud toto dané osoba nesplní, zahájí soud řízení a pokud daná osoba nesplní svou 
    povinnost ani v dodatečné lhůtě, může dostat pořádkovou pokutu nebo může dojít až k zrušení PO
\end{itemize}

\subsection{Obchodní rejstřík}

Rejstřík, do kterého se zapisují zákonem stanovené údaje, jež se týkají podnikatelů či 
organizačních složek. Je veden u krajských soudů, v Praze u městského soudu. Každý má právo do něj 
nahlížet a pořizovat si výpisy, ovšem za příslušný soudní poplatek.

\textbf{Subjekty zapisované do OR} - pouze osoby zapsané do OR mají obchodní firmu, mohou zřídit
odštěpený závod a mít prokuristu.
Zapisují se tam:
\begin{itemize}
    \item obchodní společnosti
    \item družstva a subjekty, u kterých to upravuje zákon (např. státní podniky)
    \item zahraniční FO s bydlištěm nebo PO se sídlem mimo EU podnikající na území ČR
    \item FO s vysokým obratem
\end{itemize}

Zápis obsahuje: název, sídlo, předmět činnosti, právní forma, orgány + u a.s. a s.r.o. výše základního kapitálu

\subsection{Sbírka listin}

Jedná se o součást veřejného rejstříku, obsahující důležité listiny týkající se jednotlivých 
subjektů (zakladatelská listina, společenská smlouva, účetní závěrky, podpisové vzory statutárních
orgánů)

\begin{itemize}
    \item nahlížet do ní může každý
    \item zapisované osoby sem zakládají zákonem stanovené listiny
\end{itemize}

\newpage

\section{
    Obchodní korporace, vztah NOZ a ZOK. \\ 
    \small{Podíl, vklad, základní kapitál, jednočlenná obchodní společnost}
}

\subsection{Obecně o obchodních korporacích (dle NOZ)}
\begin{itemize}
    \item jsou to právnické osoby, které mají právní osobnost
    \item korporace jsou vytvářeny jako společenství osob. Výjimka u a.s. a s.r.o., kde může být i jen
    jeden člen, mluvím pak o tzv. jednočlenné společnosti
    \item obchodní korporace mají výdělečný účel - hlavní činností je totiž podnikání, ale mohou
    být založené za účelem správy vlastního majetku (to není podnikatelská činnost) a a.s. a s.r.o. i 
    za jiným účelem
\end{itemize}

\subsection{Obchodní korporace (dle ZOK)}
\subsubsection{Obchodní společnosti}
\begin{itemize}
    \item \textbf{kapitálové}
    \begin{itemize}
        \item společnosti s ručením omezeným - s.r.o.
        \item akciové společnosti - a.s.
    \end{itemize}
    \item \textbf{osobní}
    \begin{itemize}
        \item veřejné obchodní společnosti - v.o.s
        \item komanditní společnosti - k.s.
    \end{itemize}
    \item \textbf{nadnárodní formy obchodních společností}
    \begin{itemize}
        \item Evropské společnosti - ES
        \item Evropské hospodářská zájmová sdružení - EHZS
    \end{itemize}
\end{itemize}

\subsubsection{Družstva}
\begin{itemize}
    \item družstvo - družstvo, bytové družstvo a sociální družstvo
    \item evropská družstevní společnost SCE
\end{itemize}

\subsection{Osobní společnosti}
\begin{itemize}
    \item \textbf{společníci ručí za dluhy společnosti v plném rozsahu - neomezeně} 
    (výjimkou jsou komanditisté, kteří mají postavení jako společníci u kapitálových společností)
    \item společníci jsou povinni \textbf{osobně se účastnit činnosti na společnosti}
    \item \textbf{společníci nemohou převést svůj podíl}, ale \textbf{neomezeně ručící společník
    může společnost vypovědět} (společnost pak zanikne, nebo se společníci dohodnou na jejím 
    pokračování a musí pak tomuto společníkovi vyplatit vypořádací podíl)
    \item při zániku účasti i jen jednoho ze společníků§ se společnost zásadně ruší a vstupuje 
    do likvidace
    \item \textbf{společnost netvoří povinně základní kapitál}
    \item \textbf{statutární orgány - musí být vždy společníci, vyžaduje se jednomyslnost}
\end{itemize}

\subsection{Kapitálové společnosti}

\begin{itemize}
    \item \textbf{společníci ručí za dluhy společnosti jen do výše svých nesplacených vkladů a po
    splacení neručí vůbec - omezeně}. Riskují pouze ztrátu svého vstupního vkladu, \textbf{a nebo
    neručí vůbec (a.s.)}
    \item společníci se \textbf{na činnosti společnosti osobně podílet nemusí}
    \item \textbf{podíly jsou převoditelné} - možnost převést svůj podíl, ale pouze na jiného
    společníka
    \item zánik účasti společníka na společnosti nemá na trvání společnosti žádný vliv
    \item \textbf{společnost tvoří povinně základní kapitál}, zákon předepisuje jeho minimální
    rozsah, \textbf{vklady do majetku společnosti jsou povinné}, minimální výše předepsaná zákonem
    \item \textbf{statutární orgány - členy nemusí být jen společníci} (např. profesionální
    management). Rozhodují většinovým váženým hlasováním (váha hlasu odpovídá výši vkladu společníka)
\end{itemize}

\subsection{Družstvo}
\begin{itemize}
    \item specifická forma obchodní korporace
    \item rozdíl - variabilní základní kapitál
\end{itemize}

\subsection{Vztah ZOK a NOZ}  
Zákon o obchodních korporacích je ve vztahu k občanskému zákoníku (NOZ) zvláštním zákonem.
To znamená, že občanský zákoník se použije tam, kde nelze některou otázku řešit pole ZOK -
zásada subsidiarity.

\textbf{ZOK} není nástupcem \textbf{obchodního zákoníku}, který byl zrušen, přebírá pouze tu 
část dosavadního obchodního zákoníku, která se týká \textbf{obchodních korporací}

\textbf{Výhody ZOK}: posiluje osobní zodpovědnost za správu obchodních korporací (funkčnější
ochrana třetích osob), koncept péče řádného hospodáře, liberalizace, flexibilita, systematika a 
přehlednost

\textbf{NOZ} upravuje např. prokuru, obchodní firmu, většinu smluvních typů, nekalou soutěž...

\subsection{Vklad}
\begin{itemize}
    \item \textbf{peněžní vyjádření hodnoty předmětu vkladu do základního kapitálu obchodní korporace}
    \item předmětem vkladu je věc, kterou se vkladatel (společník, budoucí společník) zavazuje vložit
    do obchodní korporace za účelem nabytí nebo zvýšení účasti v ní (vkladová povinnost)
    \item \textbf{vkladovou povinnost lze splnit}:
    \begin{itemize}
        \item \textbf{splacením} v penězích - \textbf{peněžitý vklad}
        \item \textbf{vnesením} jiné penězi ocenitelné věci - \textbf{nepeněžitý vklad}, musí být 
        oceněn znaleckým posudkem
        \begin{itemize}
            \item předmět vkladu může mít hodnotu vyšší, než je vklad - rozdíl těchto hodnot se 
            nazývá \textbf{vkladové (emisní) ažio} a není součástí základního kapitálu
            \item součet vkladu a vkladového emisního ážia se nazývá \textbf{emisní kurz}
            \item do kapitálových společností \textbf{nelze vnést nepeněžitý vklad ve formě
            práce nebo služby}
        \end{itemize}
        \item u kapitálových společností musí být před jejich vznikem \textbf{zcela vneseny
        nepeněžité vklady a musí být splaceno alespoň 30\% peněžitého vkladu a emisního ažia}
    \end{itemize}
    \item před vznikem obchodní korporace přijímá a spravuje splacené nebo vnesené předměty vkladů
    nebo jejich části společenskou smlouvou pověřený \textbf{správce vkladů}
    \begin{itemize}
        \item \textbf{peněžité vklady} - splaceny na zvláštní účet banky zřízený správcem vkladů
        \item \textbf{nepeněžité vklady} - hmotné movité nebo nemovité věci jsou předány správci
        vkladu. V ostatních případech (např. závod, pohledávka) - účinností smlouvy o vkladu
        uzavřené mezi vkladatelem a správcem vkladů
    \end{itemize}
    \item vklad není u osobních společností - není předepsaná minimální výše vkladu a tedy ani
    základní kapitál
\end{itemize}

\subsection{Základní kapitál} 

Jedná se o \textbf{souhrn všech vkladů}, majetek společnosti a veličinu pro stanovení výše
podílu každého společníka. V současné době už neplatí povinnost tvorby rezervního fondu, ale
platí povinnost dodržovat zákonné nebo smluvní postupy při \textbf{zvyšování nebo snižování ZK}

\subsection{Podíl}
Jedná se o \textbf{účast společníka v obchodní korporaci} a práva a povinnosti z této účasti
plynoucí. Každý společník může mít \textbf{pouze 1 podíl} v téže obchodní korporaci. Toto neplatí
pro kapitálové společnosti a komanditisty - ti mohou mít více podílů.

Podíl na \textbf{kapitálové společnosti}: podíly lze převádět, podíly se určují poměrem vkladů - 
podíl na ZK - výše podílu ovlivňuje hlasovací právo, právo na podíl na zisku a na likvidačním
zůstatku. Podíl může být \textbf{inkorporován do cenných papírů}, tedy u s.r.o. do kmenového
listu a u a.s. do akcií.

Přechod podílu buď smrtí nebo zánikem společníka přechází na jeho podíl v obchodní korporaci na
dědice nebo právního nástupce.

\subsection{Jednočlenná společnost}

Kapitálovou společnost může založit i jediný zakladatel, poté se zakládá obchodní korporace
\textbf{zakladatelskou listinou pořízenou ve formě veřejné listiny}. 
Kapitálová společnost může mít jediného společníka také v důsledku \textbf{soustředění všech 
podílů v jeho rukou}.
Působnost \textbf{nejvyššího orgánu} vykonává v jednočlenné společnosti její společník.

\newpage

\section{Založení a vznik, zrušení a zánik obchodních korporací. Neplatnost obchodních korporací}

\subsection{Založení a vznik}
\begin{itemize}
    \item \textbf{založení} - dochází k němu tzv. \textbf{nakladatelským právním jednáním} zakladatelů
    \item \textbf{proces předcházející vzniku obchodní korporace}:
    \begin{itemize}
        \item Obchodní korporace se zakládá \textbf{zakladatelským právním jednáním}, 
        tj. společenská smlouva, zakladatelská smlouva nebo zakladatelská listina
        \begin{itemize}
            \item společenská smlouva, kterou se zakládá kapitálová společnost, vyžaduje formu
            \textbf{veřejné listiny} , u a.s. společenská smlouva = \textbf{stanovy}
            \item u jednočlenných společností se zakládá obchodní korporace \textbf{zakladatelskou
            listinou ve formě veřejné listiny}
        \end{itemize} 
        \item má-li korporace vyvíjet podnikatelskou činnost, je podmínkou vzniku získání 
        \textbf{podnikatelského oprávnění}, pokud je založena za účelem správy vlastního majetku,
        tak toto oprávnění nepotřebuje
        \item poté se do obchodní korporace \textbf{vnáší peněžité a nepeněžité vklady}
        \item posledním krokem je \textbf{podání návrhu na zápis do obchodního rejstříku} 
        (prekluzivní lhůta je 6 měsíců ode dne jejího založení, nebude-li podán účinky zanikají)
    \end{itemize}
    \item \textbf{Obchodní korporace vzniká zápisem do obchodního rejstříku}
\end{itemize}

\textbf{Jednání jménem společnosti před jejím vznikem} - lhůta pro převzetí těchto jednání právnickou
osobou je 3 měsíce od vzniku společnosti, pak je společnost z těchto jednání zavázaná od počátku.
V opačném případě je z těchto jednání zavázána přímo osoba, která za společnost, respektive jejím
jménem, jednala

\subsection{Neplatnost obchodní korporace}

Po vzniku obchodní korporace ji soud prohlásí za neplatnou, jestliže:
\begin{itemize}
    \item společenská smlouva nebyla pořízena v předepsané formě
    \item ustanovení o výši nejnižšího splacení základního kapitálu nebyla dodržena
    \item zjistí-li nezpůsobilost všech zakládajících společníků k právnímu jednání
\end{itemize}

\subsubsection{Náležitosti nezbytná pro právní existenci právnické osoby}
Uvedení obchodní firmy, výše vkladů, celkové výše upsaného základního kapitálu, předmětu podnikání
nebo činnosti

\subsection{Zrušení a zánik obchodní korporace}

\begin{itemize}
    \item \textbf{Zrušení OK s likvidací} (viz otázka 37)
    \begin{itemize}
        \item \textbf{dobrovolná likvidace} - po uplynutí doby na kterou byla OK založena, 
        dosažením účelu, pro který byla OK založena, rozhodnutím společníků či valné hromady
        \item \textbf{nedobrovolná likvidace} - rozhodnutím soudu (společnost nemá déle než 2 roky 
        statutární orgán schopný se usnášet, společnost vyvíjí nezákonnou činnost v takové míře,
        že to závažným způsobem narušuje veřejný pořádek nebo společnost nadále nesplňuje předpoklady
        pro vznik právnické osoby)
    \end{itemize}
    \item \textbf{zrušení OK bez likvidace} - přeměna OK (viz otázka 36)
    \item \textbf{zánik} - společnost zaniká výmazem z obchodního rejstříku
\end{itemize}

\newpage

\section{Veřejná obchodní společnost, komanditní společnost}

\subsection{Veřejná obchodní společnost (v.o.s)}

\begin{itemize}
    \item v.o.s. je společnost \textbf{alespoň 2 osob (FO nebo PO)}, které se účastní na jejím 
    podnikání nebo správě jejího majetku a \textbf{ručí za její dluhy společně a nerozdílně}
    \item minimální výše základního kapitálu \textbf{není stanovena}
    \item předmětem činnosti je \textbf{podnikání nebo správa vlastního majetku}
    \item společník v.o.s. za závazky \textbf{ručí, ale neodpovídá}
    \item firma obsahuje \textbf{označení} ,,veřejná obchodní společnost" nebo zkratku v.o.s.
    \item obsahuje-li firma jméno jednoho ze společníků, stačí název ,,a spol"
\end{itemize}

\subsubsection{Založení a vznik}
\begin{itemize}
    \item společnost se zakládá uzavřením \textbf{společenské smlouvy} - písemná forma, úředně
    ověřené podpisy všech zakladatelů
    \item \textbf{vzájemné právní poměry společníků se řídí společenskou smlouvou}
    \item společenská smlouva \textbf{obsahuje} také: název, sídlo, předmět činnosti, statutární
    orgán, určení společníků
    \item společenskou smlouvu \textbf{lze měnit dohodou všech} společníků (každý společník
    má \textbf{jeden hlas}, ledaže společenská smlouva určí jinak) - má-li být změnou společenské
    smlouvy zasahováno do práv společníků, je třeba ke změně souhlasu těch společníků, do jejichž
    práv se zasahuje
    \item k rozhodování ve všech věcech společnosti je zapotřebí \textbf{souhlasu všech společníků}
    \item přisoupivší společník ručí i za \textbf{dluhy} společnosti vzniklé \textbf{před} jeho 
    přisoupením do společnosti. Může však požadovat po ostatních společnících, aby mu poskytli plnou
    náhradu za poskytnuté plnění a nahradili škody s tím spojené. Po zániku účasti ve společnosti,
    ručí společník jen za ty dluhy společnosti, které vznikly před zánikem jeho účasti.
    \item společnost vzniká \textbf{zápisem do obchodního rejstříku}
\end{itemize}

\subsubsection{Práva a povinnosti společníků}
\begin{itemize}
    \item \textbf{statutárním orgánem} společnosti jsou všichni společníci. 
    \item \textbf{společníkem nemůže být ten}, 
    \begin{itemize}
        \item na jehož majetek byl v posledních 3 letech prohlášen konkurs
        \item nebo byl návrh na zahájení insolvenčího řízení zamítnut pro nedostatek majetku
        \item a nebo byl konkurs zrušen protože je jeho majetek zcela nepostačující
    \end{itemize}
    \item \textbf{nemajetková práva} - právo podílet se na řízení a obchodním vedení společnosti -
    být statutárním orgánem společnosti, právo podat žalobu jménem společnost, právo na poskytnutí 
    informací o záležitostech společnosti (právo nahlížet do všech dokladů společnosti)
    \item \textbf{majetková práva} - právo na podíl na zisku, vypořádací podíl, podíl na likvidačním
    zůstatku
    \item \textbf{povinnosti} - společníci \textbf{nemají} vkladovou povinnost (ale můžou se k ní 
    dobrovolně zavázat ve společenské smlouvě), mají zákaz konkurence, povinnost postupovat s péčí
    řádného hospodáře, povinnost nést ztrátu společnosti
\end{itemize}

\subsubsection{Podíly společníků}
\begin{itemize}
    \item podíly společníků jsou \textbf{stejné}, pokud společenská smlouva nestanoví jinak
    \item \textbf{převod podílů se zakazuje}* - platné od 31. 12. 2020
    \item \textbf{dědic podílu}, který se nechce stát společníkem, je oprávněn svou účast
    ve společnosti vypovědět, a to ve lhůtě 3 měsíců ode dne, kdy se stal dědicem, jinak
    se k této výpovědi nepřihlíží. Během této lhůty není dědic povinen se podílet na činnosti
    společnosti
\end{itemize}

*Od 1. 1. 2021 je možné tento podíl převést se souhlasem všech společníků, bez jejich souhlasu
není převod možný. Tato změna má za cíl dosáhnout vyšší flexibility a atraktivity v.o.s.

\subsubsection{Zisk a ztráta}

\begin{itemize}
    \item zisk a ztráta se dělí mezi společníky \textbf{rovným dílem}
    \item společník, který splnil svoji vkladovou povinnost, ke které se \textbf{dobrovolné} 
    zavázal ve společenské smlouvě, má právo na podíl na zisku ve výši \textbf{25\% z částky},
    v níž splnil tuto svoji vkladovou povinnost. Pokud zisk společnosti k vyplacení této částky 
    nepostačuje, rozdělí se mezi společníky v poměru částek, v nichž splnili svou vkladovou 
    povinnost a zbylý zisk mezi společníky rovním dílem.
\end{itemize}

\subsubsection{Zánik účasti na společnosti}

Zánikem společnosti, smrtí (u FO) nebo zánikem (u PO) společníka, vystoupením ze společnosti,
výpovědí společenské smlouvy na dobu neurčitou, ...

\subsubsection{Zrušení a zánik v.o.s.}

\begin{itemize}
    \item zrušení společnosti s likvidací - \textbf{likvidace}
    \item zrušení společnosti bez likvidace - \textbf{přeměna v.o.s.}
\end{itemize}

Společnost zaniká \textbf{výmazem z obchodního rejstříku}.

\newpage

\subsection{Komanditní společnost}

Komanditní společnost je společnost, v níž alespoň \textit{jeden společník ručí za její dluhy
omezeně} - \textbf{komanditista}, a alespoň \textit{jeden společník ručí za její dluhy} -
\textbf{komplementář}, tzn. k.s. zakládají \textbf{minimálně 2 osoby} a ručí za dluhy solidárně.

\begin{itemize}
    \item předmět činnosti - \textbf{podnikání, správa vlastního majetku}
    \item firma obsahuje název ,,komanditní společnost", ,,kom. spol." nebo ,,k.s."
    \item komanditista, jehož jméno je uvedeno ve firmě musí za dluhy společnosti ručit jako 
    \textbf{komplementář}
\end{itemize}

\subsubsection{Založení a vznik}
\begin{itemize}
    \item společnost se zakládá uzavřením \textbf{společenské smlouvy} (písemka forma + úředně 
    podpisy všech zakladatelů)
    \item společenská smlouvy \textbf{obsahuje} také: určení, který ze společníků je komplementář 
    a který komanditista, výši vkladu každého komanditisty
    \item přistoupivší komplementář ručí i za dluhy společnosti vzniklé před jeho přistoupením do
    společnosti. Může však požadovat po ostatních společnících, aby mu poskytli plnou náhradu za
    poskytnuté plnění a nahradili náklady s tím spojené. Po zániku účasti ve společnosti, ručí
    komplementář jen za ty dluhy společnosti, které vznikly před zánikem jeho účasti.
    \item společnost \textbf{vzniká zápisem do obchodním rejstříku}
\end{itemize}

\subsubsection{Práva a povinnosti společníků}

\begin{itemize}
    \item \textbf{Statutární orgán} - všichni komplementáři (společenská smlouva může určit jen 
    některé z nich, nebo jen jednoho) - \textbf{komanditisté nemohou výt nikdy statutárním orgánem}
    
    \item \textbf{Společníkem nemůže být ten komplementář},
    \begin{itemize}
        \item na jehož majetek byl v posledních 3 letech prohlášen konkurs 
        \item nebo byl návrh na zahájení insolvenčního řízení zamítnut pro nedostatek majetku
        \item anebo byl konkurs zrušen protože je jeho majetek zcela nepostačující
    \end{itemize}
    
    \item \textbf{nemajetková práva společníků}
    \begin{itemize}
        \item právo k obchodnímu vedení přísluší pouze komplementářům
        \item ve věcech, které nepřísluší statutárnímu orgánu rozhodují všichni společníci, 
        přičemž hlasují komplementáři a komanditisté zvlášť
        \item právo na informace (nahlížet do všech dokladů společnosti)
    \end{itemize}
    
    \item \textbf{majetková práva společníků}
    \begin{itemize}
        \item právo na podíl na zisku, vypořádací podíl a podíl na likvidačním zůstatku
    \end{itemize}
    
    \item \textbf{povinnosti společníků}
    \begin{itemize}
        \item Komplementář je povinen postupovat s péčí řádného hospodáře
        \item lze mít spíše za to, že zákaz konkurence se vztahuje jak na komplementáře, tak na
        komanditistu (tohle je trochu sporné, dříve jen na komplementáře)
        \item komanditisté mají vkladovou povinnost \textbf{min 1 Kč}
        \item povinnost podílet se na úhradě ztráty vyplývá pouze pro komplementáře
    \end{itemize}
    
    \item \textbf{podíly komanditistů}
    \begin{itemize}
        \item \textbf{určují se podle poměru jejich vkladů}
        \item výše vypořádacího podílu komanditisty se určí podle pravidel stanovených tímto
        zákonem pro vypořádací podíl ve společnosti s ručením omezeným
        \item \textbf{za dluhy společnosti ručí komanditista s ostatními společníky společně
        a nerozdílně do výše nesplaceného vkladu podle stavu zápisu v obchodním rejstříku} -
        je-li splacen celý  vklad, neručí komanditista za dluhy KS
        \item nabytí podílu \textbf{derivativním způsobem}
        \begin{itemize}
            \item \textbf{převoditelnost podílu} - u komanditisty se použijí přiměřeně ustanovení o
            převoditelnosti podílů v s.r.o. a u komplementáře ve v.o.s
            \item \textbf{převod podílu komplementáře se zakazuje}
            \item \textbf{smrtí společníka - dědic podílu}, který se nechce stát společníkem, je
            oprávněn svou účast ve společnosti vypovědět, a to ve lhůtě 3 měsíce ode dne, kdy se 
            stal dědicem, jinak se k této výpovědi nepřihlíží - \textbf{výpovědní doba} činí tři
            měsíce a po dobu jejího běhu není dědic podílu povinen se podílet na činnosti společnosti
        \end{itemize}
    \end{itemize}
\end{itemize}

\subsubsection{Zisk a ztráta}
\begin{itemize}
    \item \textbf{zisk a ztráta} se dělí mezi společnost a komplementáře na polovinu.
    \item \textbf{komplementáři} si zisk rozdělí podle pravidel určených pro společníky v.o.s
    \item část zisku, která připadla společnosti, se po zdanění rozdělí mezi \textbf{komanditisty}
    v poměru jejich podílů
    \item ztrátu komanditisté nenesou
\end{itemize}

\subsubsection{Zrušení a zánik k.s.}
\begin{itemize}
    \item zrušení společnosti s likvidací - \textbf{likvidace}
    \item zrušení bez likvidace - \textbf{přeměna k.s.}
    \item důvodem pro zrušení k.s. není např. smrt nebo zánik komanditisty, prohlášením konkursu
    na majetek komanditisty...
    \item společnost zaniká \textbf{výmazem z obchodního rejstříku}
\end{itemize}

\subsubsection{Komanditní suma}
\begin{itemize}
    \item pokud společenská smlouva určí, že komanditisté ručí za dluhy společnosti do výše určené
    častky (,,komanditní suma"), uvede se tato částka ve společenské smlouvě
    \item nelze sjednat nižší komanditní sumu, než kolik činí vklad komanditisty
\end{itemize}

\newpage

\section{Společnost s ručením omezeným}

Společnost, za jejíž dluhy ručí společníci společně a nerozdílně do výše, v jaké nesplnili vkladové
povinnosti podle stavu zapsaného v obchodním rejstříku v době, kdy byli věřitelem vyzváni k plnění
(společníci ručí omezeně).

\begin{itemize}
    \item kapitálová společnost, \textbf{min. 1 společník} FO/PO
    \item \textbf{min. vklad je 1Kč} a minimální výše základního kapitálu je \textbf{1Kč}
    \item \textbf{předmět činnost} je podnikání, jakákoliv zákonem dovolená činnost
    \item firma obsahuje označení ,,společnost s ručením omezeným", ,,spol. s.r.o.", ,,s.r.o."
\end{itemize}

\subsection{Založení a vznik s.r.o.}
\begin{itemize}
    \item uzavřením \textbf{společenské smlouvy} - v případě, zakládá-li společnost více osob
    \item uzavřením \textbf{zakladatelské listiny} - v případě, zakládá-li společnost jedno osoba
    \item společenská smlouva musí obsahovat mimo jiné: předmět podnikání nebo účinnosti, 
    určení společníků, \textbf{počet jednatelů}, výši základního kapitálu a výši vkladu připadajících
    na podíl nebo podíly každého společníka, ...
    \item společnost vzniká \textbf{zápisem do obchodního rejstříku}
\end{itemize}

\subsection{Práva a povinnosti společníků}

\subsubsection{Společenská práva}
\begin{itemize}
    \item právo \textbf{účastnit se řízení společnosti} - prostřednictvím účasti na valné hromadě a 
    hlasování na ní
    \item právo nahlížet do všech dokladů společnosti
    \item právo podat žalobu jménem společnosti
    \item právo požádat soud o vyslovení neplatnosti usnesení valné hromady
    \item právo \textbf{ukončit účast na společnosti}
\end{itemize}

\subsubsection{Majetková práva}
\begin{itemize}
    \item právo \textbf{na podíl na zisku} - rozdělen mezi společníky v poměru jejich podílů
    \item právo na \textbf{vypořádací podíl} - podíl v poměru odpovídajícím jejich podílům
    \item právo \textbf{na podl na likvidačním zůstatku} - likvidační zůstatek se rozdělí mezi
    společníky nejprve do výše, v jaké splnili svou vkladovou povinnost. Nestačí-li likvidační
    zůstatek na toto rozdělení, podílejí se společníci na likvidačním zůstatku v poměru k výši
    svých splacených vkladů
\end{itemize}

\subsubsection{Povinnosti společníků}
\begin{itemize}
    \item \textbf{vkladová povinnost} - do 5 let ode dne vzniku společnosti, jinak úrok z prodlení
    \item \textbf{příplatková povinnost} - společníci ručí solidárně, avšak omezeně. Rozsah ručení
    je dán výší, v jaké souhrnně nesplnili vkladové povinnosti podle stavu zapsaného v obchodním
    rejstříku v době, kdy byli věřitele vyzváni k plnění
    \item \textbf{zákaz konkurence} (od 1. 1. 2021 je zákon flexibilnější, umožňuje odchýlit se 
    ve společenské smlouvě od úpravy zákonné. Je umožněno zúžit, či vyloučit všechna omezení,
    nebo určit podmínky, za nichž bude výkon zakázán, avšak v žádném případě není dotčena 
    povinnost jednat s péčí řádného hospodáře. To platí u všech kapitálových společností)
\end{itemize}

\subsection{Kmenový list}
\begin{itemize}
    \item podíl společníka může být představován \textbf{kmenovým listem}, určí-li tak společenská
    smlouva
    \item \textbf{cenný papír na řad}, umožňující snazší převod podílu ve společnosti na někoho jiného
    \item cenný papír, který se v listinné podobě převádí rubopisem a předáním (tradicí)
    \item nemůže být veřejně nabízen nebo přijat k obchodování na veřejném trhu
    \item kmenový list \textbf{nelze} vydat jako zaknihovaný cenný papír
\end{itemize}

\subsection{Podíl společníka s.r.o.}
\begin{itemize}
    \item \textbf{určuje se podle poměru jeho vkladu na tento podíl připadající k výši základního
    kapitálu}, ledaže společenská smlouva určí jinak
    \item společník může vlastnit \textbf{více podílů}, určí-li tak společenská smlouva
    \item každý společník \textbf{může převést svůj podíl} na jiného společníka anebo se souhlasem
    valné hromady i na osobu, která není společníkem
    \item \textbf{zemře-li společník}, přechází podíl na dědice - dědic podílu, který se nechce stát
    společníkem, je oprávněn svou účast ve společnosti vypovědět, a to ve lhůtě 3 měsíce ode dne,
    kdy se stal dědicem, jinak se k této výpovědi nepřihlíží
    
    \item \textbf{vznik a zánik účasti na společnosti}
    \begin{itemize}
        \item \textbf{originální} - uzavřením společenské smlouvy
        \item \textbf{derivativní} - převodem a přechodem podílu
    \end{itemize}
\end{itemize}

\subsection{Orgány společnosti}

\subsubsection{Valná hromada (obligatorní orgán)}
\begin{itemize}
    \item \textbf{nejvyšší orgán společnosti}, členy jsou všichni společníci
    \item \textbf{provádí usnesení}, jsou-li přítomni společníci, kteří mají alespoň polovinu
    všech hlasů
    \item rozhodnutí jsou přijímána \textbf{prostou většinou} přítomných společníků
    \item svolává jednatel \textbf{minimálně 1x za účetní období}
    \item \textbf{zvýšení a snížení základního kapitálu} - rozhoduje $ \frac{2}{3} $ hlasů
    \begin{itemize}
        \item zvýšení - probíhá buď převzetím vkladové povinnosti, nebo z vlastních zdrojů
        \item snížení - snížením počtu vkladů nebo snížením výše vkladů
    \end{itemize}
    \item disponuje pouze vnitřní působností, její působnost lze pouze rozšířit
\end{itemize}

\subsubsection{Jednatel (obligatorní orgán)}
\begin{itemize}
    \item \textbf{statutární orgán společnosti}
    \item jednateli přísluší \textbf{obchodní vedení společnosti} - zásadně platí, že nikdo
    není oprávněn udělovat jednateli pokyny týkající se obchodního vedení společnosti
    \item dále zajišťuje řádné vedení předepsané evidence a účetnictví, vedou seznam
    společníků
    \item musí být plně svéprávná a bezúhonná podle živnostenského zákona, nesmí u ní nastat
    překážka provozování živnosti a nesmí být vyloučena z výkonu své funkce
    \item \textbf{zastupuje s.r.o ve všech věcech}, každý jednatel jedná za společnost
    \textbf{samostatně}
    \item jeho zástupčí oprávnění je v zásadě neomezené (případné omezení společenskou smlouvou
    nemá účinky vůči třetím osobám)
    \item možnost tvořit kolektivní orgán - sbor jednatelů (jednatelstvo)
\end{itemize}

\subsubsection{Dozorčí rada (fakultativní orgán)}
\begin{itemize}
    \item její působnost je stanovena ZOK, členové voleni valnou hromadou, počet členů určuje
    společenská smlouva
    \item funkce člena dozorčí rady a jednatele je neslučitelná
\end{itemize}

\subsection{Zrušení a zánik}
\begin{itemize}
    \item kromě případů zrušení společnosti, které obecně platí pro všechny obchodní korporace,
    upravuje ZOK dva zvláštní případy zrušení s.r.o.:
    \begin{itemize}
        \item \textbf{všichni společníci se dohodnou o zrušení společnosti} - dohoda má formu 
        veřejné listiny
        \item \textbf{společník se může také domáhat zrušení společnosti u soudu} z důvodu
        určených společenskou smlouvou
    \end{itemize}
    \item zrušení společnosti 
    \begin{itemize}
        \item s likvidací - \textbf{likvidace}
        \item bez likvidace - \textbf{přeměna s.r.o.}
    \end{itemize}
    \item společnost zaniká \textbf{výmazem z obchodního rejstříku}
\end{itemize}

\subsection{Přeměny s.r.o.}
\begin{itemize}
    \item \textbf{fúze} - sloučením nebo splynutím společností s ručením omezeným,
    připuštěno sloučení nebo splynutí s akciovou společností, nástupnická společnost může 
    být akciová nebo s ručením omezením.
    \item \textbf{převod jmění na společníka} - jen v případě, že jeho podíl představuje 
    nejméně 90\%  základního kapitálu (a současně 90\% hlasovacích práv).
    \item \textbf{změna právní formy} - možnost změny na veřejnou, komanditní, akciovou 
    společnost i družstvo
\end{itemize}

\newpage

\section{Akciová společnost}

Obchodní společnost, jejíž základní kapitál je rozvržen na určitý počet akcií. 
Obecně platí, že ,,co není zakázáno, je povoleno". Akciová společnost může být založena
jedním nebo více zakladateli (právnické a fyzické osoby). Firma musí nést označení 
,,akciová společnost", ,,akc. spol." nebo ,,a.s."

\subsection{Akcie} 
Jedná se o cenný papír (zaknihovaný, nebo nikoliv), s nímž jsou spojená práva a povinnosti akcionáře
jako společníka a.s. podílet se na řízení společnosti, jejím zisku a na likvidačním zůstatku, při
jejím zrušení s likvidací. Jedná se o podíl akcionáře na akciové společnosti a na jejím základním
kapitálu. 

Společnost vydává buď akcie se \textbf{jmenovitou hodnotou} nebo (xor) tzv. \textbf{kusové akcie}.
Kusové akcie nemají jmenovitou hodnotu a jejich účetní hodnota se určuje počtem vydaných kusových
akcií.

\subsubsection{Náležitosti akcií vydávanyých jako CP}
Je nutné označení, že jde o akcii, identifikace společnosti, která akcii vydala, jmenovitou 
hodnotu nebo označení ,,kusová akcie", identifikace akcionáře (u akcií na jméno), údaje o druhu
akcie.
Více viz otázka 34. % TODO: add section links

\subsubsection{Druhy akcií}
\begin{itemize}
    \item \textbf{nesplacená akcie} - práva a povinnosti akcionáře do doby splacení emisního
    kurzu akcií (určí-li tak stanovy, můžou být tyto práva a povinnosti spojena se 
    \textbf{zatímním listem} - cenný papír představující nesplacenou akcii)
    \item \textbf{nevydaná akcie} - práva a povinnosti akcionáře po splacení emisního kurzu 
    akcií, nebyl-li vydán cenný papír nebo zaknihovaný cenný papír představující účast na společnosti
\end{itemize}

\subsection{Seznam akcionářů}
Akcie na jméno se zapisuje do seznamu akcionářů, který vede společnost (vydala-li společnost 
zaknihované akcie, mohou stanovy určit, že seznam akcionářů je nahrazen evidencí zaknihovaných
cenných papírů). Jako \textbf{účastnické cenné papíry} se označují CP vydané akciovou společností,
se kterými je spojen podíl na základním kapitálu nebo hlasovacích právech této společnosti (zejména
\textbf{akcie} a \textbf{zatímní listy})

\subsection{Založení a vznik}
\begin{itemize}
    \item k založení se \textbf{vyžaduje přijetí stanov}, které mají formu \textbf{veřejné listiny} 
    (notářského zápisu)
    \item \textbf{založení společnosti je účinné}, splatí-li každý zakladatel případné emisní ážio a
    v souhrnu alespoň 30\% jmenovité nebo účetní hodnoty upsaných akcií v době určené ve stanovách a
    na účet banky určený ve stanovách 
    \item Akciová společnost vzniká \textbf{dnem zápisu do obchodního rejstříku}
\end{itemize}

\subsection{Orgány společnosti}

\subsubsection{Dualistický systém (vnitřní struktury) a.s.}
\begin{itemize}
    \item \textbf{nejvyšší orgán: valná hromada}
    \begin{itemize}
        \item kolektivní orgán, který tvoří všichni akcionáři
        \item svolaná alespoň 1x za účetní období
        \item rozhoduje o základních otázkách společnosti - např. změna stanov, změna výše ZK,
        schvaluje účetní závěrky, rozdělelní zisku
        \item rozhoduje \textbf{usnesením}
        \item její působnost je taxativně vymezená zákonem - lze ji pouze rozšířit
        \item \textbf{je usnášeníschopná}, pokud jsou přítomni akcionáři vlastnící akcie,
        jejichž jmenovitá či účetní hodnota přesáhne 30\% základního kapitálu - tzv. \textbf{kvorum}
        \item z jednání valné hromady se vyhotovuje zápis a seznam přítomných
    \end{itemize}
    
    \item \textbf{statutární orgán: představenstvo}
    \begin{itemize}
        \item rozhoduje o každodenních otázkách, zastupuje a.s. ve všech záležitostech, 
        obchodní vedení
        \item má tzv. \textbf{zbytkovou působnost} - veškerá působnost, kterou stanovy či právní
        předpisy nesvěří do rukou jiného orgánu a.s.
        \item má zásadně 3 členy, členové mají jednoleté funkční období
    \end{itemize}
    
    \item \textbf{kontrolní orgán: dozorčí rada}
    \begin{itemize}
        \item počet jejích členů musí být dělitelný třemi (dvě třetiny dozorčí rady volí valná
        hromada, jednu třetinu zaměstnanci společnosti).
        \item 3-leté funkční období
        \item kontrolní činnost - dohlíží na řádný výkon představenstva a na činnost společnosti
        jak takové, oprávněna nahlížet do všech dokladů a záznamů
    \end{itemize}
\end{itemize}

\subsubsection{Monoistický systém (vnitřní struktury) a.s.}
\begin{itemize}
    \item \textbf{nejvyšší orgán: valná hromada}
    \item \textbf{statutární orgán: správní rada} - od 31. 12. 2020 je statutárním orgánem 
    \textbf{statutární ředitel}
    \item \textbf{správní (kontrolní) orgán: správní rada}
    \begin{itemize}
        \item má ze zákona 3 členy, ale může existovat i jako unipersonální orgán
        \item členy volí a odvolává valná hromada, správní rada má svého předsedu
        \item do jejího působení patří vše, co zákon svěřuje do působnosti představenstva a 
        dozorčí rady s výjimkou běžného obchodního vedení a zastupování společnosti
    \end{itemize}
\end{itemize}

\subsection{Práva a povinnosti akcionářů}

\subsubsection{Majetková práva}
\begin{itemize}
    \item právo na \textbf{podíl na zisku}
    \item právo na \textbf{likvidačním zůstatku} - akcionáři se dělí o likvidační zůstatek v poměru
    odpovídajícím splacené jmenovité nebo účetní hodnotě jejich akcií
    \item právo \textbf{účasti na zvyšování základního kapitálu} - právo na upisování nebo koupi
    akcií
    \item právo na \textbf{odkoupení akcií} za podmínek stanovených zákonem
\end{itemize}

\subsubsection{Nemajetková práva}
\begin{itemize}
    \item právo na rovné zacházení - \textbf{společnost zachází za stejných podmínek se všemi
    akcionáři stejně}
    \item právo na \textbf{vydání akcie nebo zatímního listu} při splacení emisního kurzu akcie
    \item \textbf{právo podílet se na řízení společnosti} - právo účastnit se valné hromady,
    právo uplatňovat návrhy a protinávrhy, právo na vysvětlení, hlasovací právo
    \begin{itemize}
        \item \textbf{korespondenční hlasování} - společníci odevzdají své hlasy písemně před
        konáním valné hromady
        \item \textbf{rozhodování per rollam} - rozhodování mimo valnou hromadu (ta se nekoná a
        každý akcionář se vyjádří k rozhodnutí v určité lhůtě)
        \item \textbf{kumulativní hlasování} - můžou se tak volit členové orgánů společnosti
    \end{itemize}
    \item \textbf{práva kvalifikovaných akcionářů}
    \begin{itemize}
        \item právo požadovat svolání valné hromady
        \item právo podat akcionářskou žalobu
    \end{itemize}
\end{itemize}

\subsubsection{Povinnosti}
\begin{itemize}
    \item vkladová povinnost - povinnost splatit \textbf{emisní kurz akcií} - částka, za kterou
    vydává společnost akcie (jmenovitá hodnota akcie + emisní ážio)
    \item povinnost \textbf{loajality} - chovat se čestně a zachovat vnitřní řád společnosti
\end{itemize}

\subsection{Zrušení a zánik}
\begin{itemize}
    \item všichni společníci se dohodnout o zrušení společnosti - dohoda má formu
    \textbf{veřejné listiny}
    \item \textbf{společník se také může domáhat} zrušení společnosti u soudu z důvodů
    určených společenskou smlouvou
    \item zrušení společnosti buď \textbf{s likvidací} nebo bez likvidace (\textbf{přeměna a.s.})
    \item společnost zaniká \textbf{výmazem z obchodního rejstříku}
\end{itemize}

\newpage

\section{Cenné papíry, akcie, kmenový list \\ \small{pojem, forma, druhy, vydávání}}

\textbf{Cenný papír} je listina, se kterou je právo spojeno takovým způsobem, že je po vydání
cenného papíru nezle bez této listiny uplatnit, ani převést (výjimkou je třeba akcie na jméno)

\subsection{Typy cenných papírů}

\subsubsection{Akcie}
Jedná se o cenný papír nebo zaknihovaný cenný papír, s nímž jsou spojena práva a povinnosti akcionáře
jako společníka a.s. podílet se na řízení společnosti, jejím zisku a na likvidačním zůstatku při
jejím zrušení s likvidací.

\subsubsection{Podílové listy}
Jedná se o majetkové cenné papíry neboli doklady o vlastnictví nebo spoluvlastnictví majetku v 
otevřeném nebo uzavřeném podílovém fondu

\subsubsection{Dluhopisy}
Cenné papíry na řad nebo zaknihované cenné papíry, s nímž je spojeno právo na splácení určité dlužné 
částky

\subsubsection{Šeky}
Bankovní formuláře, ze kterých se po vyplnění předepsaných náležitostí stává platební prostředek nebo
cenný papír

\subsubsection{Další cenné papíry}
Např. kupóny, směnky, náložné listy, opční listy...

\subsection{Dělení z hlediska podoby}
\begin{itemize}
    \item \textbf{Listinný cenný papír} - jedná se o všechny fyzické listiny
    \item \textbf{Zaknihovaný cenný papír} -  jedná se o zápis v evidenci. Pro uplatňování 
    práv spojených se ZCP slouží výpisy z evidence ZCP. Evidence se vede na majetkových účtech
    vlastníka nebo zákazníka 
    \item \textbf{Imobilizovaný cenný papír} - jedná se o jakýsi ,,mezikrok" mezi CP a ZCP.
    Je to listinný cenný papír, který byl uložen do hromadné úschovy emitentem a je o něm 
    vedena evidence.
\end{itemize}


\subsection{Dělení z hlediska formy cenného papíru}
\begin{itemize}
    \item \textbf{Na doručitele (na majitele)} - vlastníkem je ten, kdo ho předloží,
    není-li prokázán opak
    \item \textbf{Na řad} - cenný papír, který se v listinné podobě převádí rubopisem, v němž se 
    uvede jednoznačná identifikace nabyvatele
    \item \textbf{Na jméno} - vlastníkem cenného papíru je ten, na kohož jméno byl cenný papír vydán
\end{itemize}

\subsection{Dělení z hlediska druhu}
\begin{itemize}
    \item \textbf{kmenové akcie} - nejběžnější, s právem na podíl na zisku a likvidačním zůstatku,
    na účast a hlasování na valné hromadě
    \item \textbf{prioritní akcie} - akcie, se kterou jsou spojena prioritní práva týkající se 
    podílu na zisku nebo na likvidačním zůstatku. Jsou vydány bez hlasovacích práv
\end{itemize}

\textbf{ZOK zakazuje tzv. úrokové akcie}

\subsection{Kmenový list}
\begin{itemize}
    \item může představovat \textbf{podíl společníka s.r.o.}
    \item \textbf{cenný papír na řad} - cenný papír, který se v listinné podobě převádí pouze
    rubopisem a předáním tradicí
    \item Umožňuje snazší převo podílu ve společnosti na někoho jiného
    \item \textbf{Kmenový list nelze vydat jako zaknihovaný cenný papír}
\end{itemize}

\newpage

\section{Orgány obchodních korporací, smlouva o výkonu funkce, péče řádného hospodáře}

\subsection{Orgány obchodních společností}

\begin{itemize}
    \item \textbf{kolektivní orgán} - zahrnuje více členů
    \item \textbf{individuální orgán} - zahrnuje jednoho člena
    \item kolektivní orgán \textbf{rozhoduje většinou hlasů zúčastněných členů} za podmínky
    přítomnosti nebo jiné účast (např. videokonference, korespondenční hlasování) 
    \textbf{většiny všech členů} daného orgánu (tzv. \textbf{kvorum})
    \item  \textbf{zakladatelské právní jednání} však může stanovit vyšší počet potřebný pro přijetí
    usnesení (tzv. \textbf{kvalifikovaná většina}) i pro prezenci či účast při rozhodování
    \item \textbf{volené orgány} jsou orgány, které jsou obsazovány na základě volby, členy volených
    orgánů volí zásadně nejvyšší orgán obchodní korporace
    \item členové volných orgánu mohou být \textbf{kdykoliv odvoláni}, pokud o tom rozhodne ten,
    v jehož působnosti je jejich volba (odvolání nemusí být nijak zdůvodněno)
    \item člen voleného orgánu může ze své funkce \textbf{odstoupit}, nesmí tak však učinit v době,
    která není pro obchodní korporaci vhodná
    \item v případě kapitálových obchodních společností zákon výslovně \textbf{zakazuje komukoliv
    udělovat pokyny příslušnému orgánu v oblasti obchodního vedení} 
\end{itemize}

\subsection{Typy orgánů}

\subsubsection{Obligatorní orgány}
Orgány, jejichž existenci přímo vyžaduje zákon.

\subsubsection{Fakultativní orgány}
Orgány, které vznikají z vůle společníků, i když zákon jejich zřízení nevyžaduje.

\textbf{Kontrolní orgán} sleduje a kontroluje činnost statutárních orgánů, dohlíží na uskutečňování
podnikatelské činnosti společnosti a kontroluje hospodaření společnosti. Obligatorním orgánem je jen
u akciové společnosti a družstva.

\newpage

\subsection{Orgány jednotlivých obchodních korporací}

\begin{itemize}
    \item \textbf{nejvyšší orgán} - jmenuje a odvolává statutární/kontrolní orgány, rozhoduje o 
    základním kapitálu, přeměnách obchodní korporace, změny ve společenské smlouvě
    \item \textbf{statutární orgán} - rozhoduje o všech záležitostech, které nejsou svěřeny jinému
    orgánu (tzv. zbytková působnost), vykonává obchodní vedení a zastupuje obchodní korporaci
    \item \textbf{výbor pro audit} - povinný orgán pro společnosti, které naplňují definici tzv.
    subjektu veřejného zájmu podle zákona o auditorech
\end{itemize}

\subsubsection{Veřejná obchodní společnost}
\begin{itemize}
    \item \textbf{nejvyšší orgán}: všichni společníci
    \item \textbf{statutární orgán}: každý společník
\end{itemize}

\subsubsection{Komanditní společnost}
\begin{itemize}
    \item \textbf{nejvyšší orgán}: všichni společníci
    \item \textbf{statutární orgán}: všichni komplementáři
\end{itemize}

\subsubsection{Společnost s ručením omezeným}
\begin{itemize}
    \item \textbf{nejvyšší orgán}: valná hromada
    \item \textbf{statutární orgán}: jednatel, případně jednatelé
\end{itemize}

\subsubsection{Akciová společnost}
\begin{itemize}
    \item \textbf{dualistická struktura}
    \begin{itemize}
        \item \textbf{nejvyšší orgán}: valná hromada
        \item \textbf{statutární orgán}: představenstvo (kolektivní orgán)
        \item \textbf{kontrolní orgán}: dozorčí rada - \textbf{povinná}
    \end{itemize}
    \item \textbf{monistická struktura}
    \begin{itemize}
        \item \textbf{nejvyšší orgán}: valná hromada
        \item \textbf{statutární orgán}: statutární ředitel
        \item \textbf{správní (kontrolní) orgán}: správní rada
    \end{itemize}
\end{itemize}

\subsubsection{Družstvo}
\begin{itemize}
    \item \textbf{nejvyšší orgán}: členská schůze
    \item \textbf{statutární orgán}: představenstvo
    \item \textbf{kontrolní orgán}: kontrolní komise
\end{itemize}

\subsection{Péče řádného hospodáře}
\begin{itemize}
    \item spočívá v jednání při výkonu funkce s \textbf{nezbytnou loajalitou, s potřebnými 
    znalostmi a pečlivostí}
    \item za porušení této péče se výslovně považuje i situace, kdy určitá osoba přijme funkci, 
    byť si musela být vědoma, že na ni znalostmi nebo z jiných důvodů nestačí - \textbf{jedná
    nedbale a je povinna nahradit obchodní korporaci škodu} vzniklou při výkonu této funkce
    \item ZOK bere v úvahu \textbf{pravidlo podnikatelského úsudku} - pečlivě a s potřebnými 
    znalostmi jedná ten, kdo mohl při podnikatelském rozhodování v dobré víře rozumně předpokládat,
    že jedná informovaně a v obhajitelném zájmu obchodní korporace
    \item posuzuje-li soud, zda člen voleného orgánu porušil povinnost jednat s péčí řádného
    hospodáře, \textbf{nese člen důkazní břemeno} tj. on je ten, kdo musí dokázat svou nevinnu
\end{itemize}

\subsection{Smlouva o výkonu funkce}
\begin{itemize}
    \item zvláštní smlouva příkazního typu, smlouva člena orgánu s obchodní korporací
    \item \textbf{výkon funkce člena orgánu obchodní korporace zásadně nemá charakter
    pracovního poměru}
    \item schvaluje jí ten orgán, který člena orgánu zvolil
    \item jsou zde obsaženy podmínky výkonu funkce, dále obsahuje náležitosti jako je odměna za výkon,
    práva a povinnosti člena
\end{itemize}

\section{Zrušení a přeměny obchodních korporací}

\subsection{Zrušení obchodní korporace bez likvidace}


\begin{itemize}
    \item \textbf{Právní účinky} fúze i rozdělení nastávají \textbf{zápisem do obchodního rejstříku} 
    avšak \textbf{účetní účinky} nastávají k \textbf{rozhodnému dnu}, který nastává dříve.
    \item ze společníků obchodní korporace, která v procesu fúze nebo rozdělení zanikne,
    \textbf{se stávají společníky nástupnické korporace} (i u odštěpení mají společníci právo na
    účast na nově vzniklé korporaci)
    \item při fúzi \textbf{přecházení dluhy} zanikající korporace na nástupnickou korporaci. Při 
    rozdělení se dluhy ,,rozdělí" mezi nástupnické korporace.
    \item je připuštěno \textbf{,,křížení"} různých forem korporací je \textbf{připuštěna pouze 
    vzájemně} mezi osobními (v.o.s a k.s.) a kapitálovými (s.r.o. a a.s.) obchodními společnostmi
\end{itemize}

\subsubsection{Fúze}
\begin{itemize}
    \item \textbf{sloučením} - při \textbf{sloučení} nejméně jedna ze zúčastněných osob zaniká, 
    práva a povinnosti zanikajících osob jako na nástupnickou právnickou osobu. A + B = B
    \item \textbf{splynutím} - při \textbf{splynutí} zanikají všechny zúčastněné osoby a na jejich 
    místě vzniká nová právnická osoba jako osoba nástupnická. Na tu pak přechází práva a povinnosti
    všech zanikajících osob. A + B = C
\end{itemize}

\subsubsection{Rozdělení (včetně přeshraničního)}
\begin{itemize}
    \item \textbf{Odštěpením} - z korporace Y se vydělí určitá část jejího jmění, která přejde na 
    nástupnickou korporaci Z, která zpravidla existuje již před rozdělením, nebo může vzniknout
    až v rámci rozdělení. Korporace Y následně nadále pokračuje ve své činnosti.
    
    \item \textbf{Rozštěpením} - korporace X zanikne a její jmění přejde na korporace Y a Z, které
    již existují, nebo vzniknou v rámci fúze.
    
    Obchodní korporace se mohou slučovat a dělit pouze v téže právní formě, nebo pouze vzájemně
    mezi osobními a kapitálovými společnostmi
\end{itemize}

\subsubsection{Převod jmění na společníka}
Jedná se o speciální případ \textbf{fúze mezi společností a jejím společníkem}.

\begin{itemize}
    \item společnost zaniká a její jmění přechází na společníka, jakožto jejího nástupníka
    \item \textbf{u osobních společností}, pouze pokud společnost \textbf{již jiného společníka
    nemá}, převod jmění na společníka umožní zachovat provoz závodu nebo se vyhnout administrativně
    náročnějšímu procesu likvidace.
    \item \textbf{od 1. 1. 2021 je možné převést podíl i v.o.s se souhlasem všech společníků}
    \item u \textbf{kapitálových společností} je možné převést podíl jen na společníka, jehož
    \textbf{podíl představuje nejméně 90\% na základním kapitálu a hlasovacích právech}. Ostatním
    společníkům tento společník musí vyplatí přiměřené vypořádání.
    \item \textbf{u družstev se převod jmění na společníka zakazuje}
\end{itemize}

\subsubsection{Změna právní formy}
\begin{itemize}
    \item \textbf{korporace nezaniká}, ale dochází ke změně její právní formy, tedy \textbf{vnitřní
    právní struktury}
    \item každá korporace může změnit svoji právní formu na jakoukoliv z ostatních forem obchodních
    korporací
    \item \textbf{změna právní formy nemůže vést ke snížení rozsahu ručení společníků} za dluhy
    korporace vzniklé před účinností přemněny právní formy
\end{itemize}

\subsubsection{Přeshraniční přemístění sídla}
\begin{itemize}
    \item zahraniční PO přemisťující své sídlo do ČR si \textbf{zvolí prání formu české právnické
    osoby} a musí se \textbf{přizpůsobit českému právu}
    \item české PO přemisťující své sídlo do zahraničí se zase musí přizpůsobit \textbf{právu země,
    do které se stěhují}
\end{itemize}

\subsection{Zrušení obchodních korporací s likvidací}
\begin{itemize}
    \item \textbf{společnost se zruší s likvidací} - práva a povinnosti korporace zanikají spolu s ní
    \begin{itemize}
        \item \textbf{dobrovolná likvidace} - po uplynutí doby, na kterou byla OK založena, dosažením
        účelu, pro který byla OK založena, rozhodnutím společníků či valné hromady
        \item \textbf{nedobrovolná likvidace} - rozhodnutím soudu (společnost nemá déle než 2 roky 
        statutární orgán schopný se usnášet, společnost vyvíjí nezákonnou činnost v takové míře,
        že to závažným způsobem narušuje veřejný pořádek, společnost nadále nesplňuje předpoklady 
        vyžadované pro vznik PO zákonem)
    \end{itemize}
    \item poté dochází k \textbf{vypořádání právních vztahů, zpeněžení majetku korporace a rozdělení
    likvidačního zůstatku společníkům}
\end{itemize}

\subsubsection{Likvidace}
\begin{itemize}
    \item při vstupu do likvidace musí být jmenován \textbf{likvidátor}, který jako \textbf{zvláštní
    orgán korporace přebírá působnost statutárního orgánu}, kterou vykonává v rozsahu a takovým 
    způsobem, aby bylo dosaženo cíle likvidace
    \item likvidátor musí oznámit všem věřitelům obchodní korporace její \textbf{vstup do likvidace}
    a vyzve je, aby přihlásili své pohledávky
    \item ze \textbf{zpeněženého majetku} (likvidační podstaty) se postupně hradí náklady likvidace,
    pohledávky zaměstnanců, pohledávky ostatních věřitelů
    
    \item likvidace předlužené korporace - korporace má majetek, který nestačí k úhradě všech
    závazků - korporace je v úpadku a likvidátor musí podat \textbf{insolvenční návrh}
    
    \item likvidace nepředlužené korporace - korporace má majetek, který stačí k úhradě všech
    \begin{itemize}
        \item likvidace končí \textbf{rozdělením likvidačního zůstatku mezi společníky} (tomu 
        předchází vyhotovení konečné zprávy o průběhu likvidace a konečné účetní závěrky)
        \item \textbf{po vyplacení likvidačního zůstatku podá likvidátor návrh na výmaz obchodní
        korporace z obchodním rejstříku, a tím obchodní korporace zaniká}
    \end{itemize}
\end{itemize}

\newpage

\section{Relativní majetková práva \\ \small{obecná charakteristika, pohledávka, dluh}}

\subsection{Relativní majetková práva (závazky)}
\begin{itemize}
    \item působí jenom mezi stranami závazku
    \item vznik ze smlouvy, z porušení smlouvy či zákona
    \item relativní charakter závazku spočívá v tom, že vzniká mezi dvěma nebo více určitými
    subjekty (věřitel a dlužník)
\end{itemize}

\subsection{Dluh}
\begin{itemize}
    \item předmět vztahu mezi věřitelem a dlužníkem, který zahrnuje oprávnění věřitele a 
    povinnost dlužníka, aby dluh vyrovnal
\end{itemize}

\subsection{Pohledávka}
\begin{itemize}
    \item oprávnění věřitele požadovat od dlužníka vyrovnání dluhu, tedy takzvané plnění
    \item pohledávka může být peněžitá nebo nepeněžitá
    \item jedná se o majetkovou hodnotu, kterou lze až na výjimky převést na jiného 
    (postoupení pohledávky) a která, nezanikne-li smrtí věřitele, může být předmětem dědění
\end{itemize}

\subsection{Charakteristické rysy závazků}
\begin{itemize}
    \item individuálně určený počet subjektů
    \item relativnost závazků - vzájemnost práv a povinností
    \item majetkový charakter - dluh (plnění) má majetkovou povahu (dá se hospodářsky ocenit)
    \item dispozitivnost - obsah závazku budou především určovat projevy vůle samotných subjektů,
    nejen právní normy
\end{itemize}

\subsection{Druhy závazků}
\begin{itemize}
    \item \textbf{podle počtu subjektů}
    \begin{itemize}
        \item jednoduché
        \item společné - dílčí, solidární, nedílné...
    \end{itemize}
    \item \textbf{podle původu vzniku}
    \begin{itemize}
        \item kauzální - nejčastěji důvod ekonomický
        \item abstraktní - nevyjadřují důvod vzniku
    \end{itemize}
    
    \item \textbf{podle předmětu a způsobu plnění}
    \begin{itemize}
        \item závazky s plněním (dluhem) určeným - jednotlivě, podle druhu, alternativně
        \item závazky vzájemné a jednostranné
        \item závazky s plněním jednorázovým, opakovaným a nepřetržitým
        \item závazky ve prospěch třetího (např. pojistná smlouva uzavřená ve prospěch třetí osoby
        - pojištěného)
    \end{itemize}
    
    \item \textbf{podle způsobu jejich vzniku}
    \begin{itemize}
        \item závazky z právního jednání
        \item závazky z protiprávního jednání (z deliktů)
        \item závazky vznikající na základě jiných právních skutečností
    \end{itemize}
\end{itemize}

\newpage

\section{Vznik závazků, postup při uzavírání smluv}

Obchodní závazky mohou vznikat \textbf{ze smluv, z protiprávního jednání, či jiné právní skutečnosti},
která je k tomu podle právního řádu způsobilá

\subsection{Smlouva}
\begin{itemize}
    \item smlouva představuje nejčastější důvod vzniku obchodních závazků
    \item smlouva je svobodný projev vůle o vzniku závazku mezi stranami i o konkrétním obsahu
    této smlouvy
    \item smlouva je uzavřena, jakmile si strany ujednali její obsah
    \item \textbf{forma}
    \begin{itemize}
        \item \textbf{písemná} - nutná stanoví-li to zákon. Pokud alespoň jedna strana vyžaduje
        písemnou formu, použije se písemná forma
        \item \textbf{ústní} - není-li nutná písemná smlouva (kupní smlouva, smlouva o dílo, ...)
    \end{itemize}
    \item \textbf{prvky smlouvy (závazku = právního vztahu)}"
    \begin{itemize}
        \item subjekty
        \item obsah - práva a povinnosti, obchodní podmínky
        \item objekt - dát, vykonat, zdržet se, strpět
    \end{itemize}
\end{itemize}

\subsection{Uzavírání smluv}
\begin{enumerate}
    \item \textbf{Nabídka (oferta) - návrh smlouvy}
    \begin{itemize}
        \item \textbf{jedna strana učiní nabídku a druhá jí akceptuje} - smlouva je dvoustranný úkon
        \item \textbf{nabídka musí obsahovat}
        \begin{itemize}
            \item podstatné náležitosti smlouvy
            \item musí z ní plynout vůle navrhovatele být smlouvou vázán v případě jejího přijetí
            \item musí být určena jedné nebo více konkrétním osobám
        \end{itemize}
        \item konkrétní délka lhůty pro přijetí nabídky (\textbf{akceptační lhůta}) ze zákona 
        nevyplývá, ale měla by být obsažena v nabídce
        \item návrh působí od doby, kdy dojde osobě, které je určen (\textbf{tzv. teorie dojití})
        \item nabídku může strana \textbf{zrušit}, pokud zrušovací návrh dojde před nebo současně 
        s nabídkou
        \item nabídka může být \textbf{odvolána}, ale jen pokud odvolání dojde před tím, než ji
        druhá strana akceptuje (existuje i neodvolatelná nabídka)
        \item nabídka \textbf{zaniká} uplynutím akceptační lhůty, odmítnutím, smrtí/zánikem některé
        ze smluvních stran
    \end{itemize}
    \item \textbf{Přijetí (akceptace) nabídky a uzavření smlouvy}
    \begin{itemize}
        \item \textbf{uzavření smlouvy} se stanoví okamžikem, kdy přijetí nabídky nabývá účinnosti
        \item \textbf{včasné přijetí} - smlouva vzniká
        \item \textbf{opožděné přijetí} - k přijetí smlouvy je potřeba souhlas navrhovatele
        \item obsahuje-li přijetí nabídky \textbf{dodatky, výhrady, omezení nebo jiné změny},
        přestavuje odmítnutí nabídky a považuje se za nabídku novou + odpověď \textbf{s dodatkem 
        nebo odchylkou, která podstatně nemění podmínky nabídky} je považována za přijetí nabídky
    \end{itemize}
\end{enumerate}

\subsection{Zvláštní způsoby uzavírání smluv}
\begin{itemize}
    \item jedná se o \textbf{veřejnou soutěž o nejvhodnější nabídku, veřejnou nabídku a smlouvu o 
    smlouvě budoucí}
    \item NOZ pojímá \textbf{dražbu} jako obecný způsob uzavření smlouvy, kdy tato je uzavřena
    \textbf{příklepem} a učiněná nabídka se zruší podáním nabídky vyšší
    \item veřejná soutěž o nejvhodnější nabídku a veřejná nabídka mají společné to, že se obracejí
    na neurčitý okruh osob. rozdílné mají to, kde je iniciátorem uzavření smlouvy, v prvním případě
    je to její příjemce, ve druhém její navrhovatel
\end{itemize}

\subsubsection{veřejná soutěž o nejvhodnější nabídku}
\begin{itemize}
    \item uzavření maximálně výhodné smlouvy
    \item vyhlašovatel vyhlásí veřejnou soutěž, čímž učiní výzvu k podání nabídek. Vyhlašovatel
    pak vybere a přijme nejvhodnější nabídku a tímto přijetím je uzavřena smlouva.
    \item je zahájena výzvou k podání nabídek, je požadováno, aby byl vymezen předmět, způsob a
    lhůta, ve které lze návrhy podat
\end{itemize}

\subsubsection{Veřejná nabídka}
\begin{itemize}
    \item právní jednání, které směřuje k více osobám
    \item navrhovatel se obrací na neurčité osoby s návrhem na uzavření smlouvy (nabídkou)
    \item smlouva je uzavřená s tím, kdo nabídku přijme jako první
\end{itemize}

\subsubsection{Zadávání veřejných zakázek}
\begin{itemize}
    \item mnoho společných rysů s veřejnou soutěží
    \item veřejnou zakázkou je zakázka realizovaná na základě smlouvy mezi zadavatelem a jedním
    či více dodavateli, jejímž předmětem je úplatné poskytnutí dodávek či služeb, nebo úplatné
    provedení stavebních prací
    \item \textbf{zadavatelé}
    \begin{itemize}
        \item \textbf{veřejný zadavatel} - především stát (organizační složky státu)
        \item \textbf{dotovaný zadavatel} - PO nebo FO, zadaná zakázka je hrazena z 50\% z 
        veřejných zdrojů a ze soukromých zdrojů
        \item \textbf{sektorový zadavatel} - např. plynárenství, teplárenství, elektroenergetika
    \end{itemize}
    \item \textbf{druhy zakázek podle hodnoty}
    \begin{itemize}
        \item podlimitní
        \item nadlimitní
        \item zakázky malého rozsahu
    \end{itemize}
    \item \textbf{druhy zakázek podle předmětu}
    \begin{itemize}
        \item veřejné zakázky na dodávky
        \item zakázky na stavební práce
        \item zakázky na služby
    \end{itemize}
    \item \textbf{postup řízení}
    \begin{enumerate}
        \item zahájení
        \item nabídka - návrh smlouvy
        \item vyhodnocení
        \item uzavření písemné smlouvy
    \end{enumerate}
    \item dohled nad zadáváním veřejných zakázek - úřad pro ochranu hospodářské soutěže, může 
    dojít ke zrušení zadání zakázky.
\end{itemize}

\subsubsection{smlouva o smlouvě budoucí}
\begin{itemize}
    \item není to typ smlouvy, ale způsob uzavření smlouvy
    \item povinnost \textbf{zavázané strany} uzavřít smlouvu bez zbytečného odkladu poté, co
    k tomu byla vyzvána \textbf{stranou oprávněnou}
    \item vyzvání ve lhůtě 1 rok, jinak účinky zanikají
\end{itemize}

\newpage

\section{Změny závazků}
\begin{itemize}
    \item závazky se \textbf{mohou v průběhu svého trvání změnit}, aniž by došlo k jejich zániku
    (změna majetku, obsahu, práva a povinností stran)
    \item vyloučena je pouze tam, kde jsou druh nebo pohledávka vázány na určitou osobu, nebo tam,
    kde by změna závazku znamenala jeho zánik, nebo nahrazení závazkem novým
\end{itemize}

\subsection{Změny v subjektech}
\subsubsection{Postoupení pohledávky nebo souboru pohledávek}
\begin{itemize}
    \item změny v osobě věřitele
    \item dohoda mezi věřitele (postupitelem) a třetí osobou (postupníkem), na jejímž základě
    přechází na třetí osobu spolu s postoupenou pohledávkou i její příslušenství a všechna
    práva s ní spojená, souhlasu dlužníka není třeba, ale musí být informován
\end{itemize}

\subsubsection{Převzetí dluhu}
\begin{itemize}
    \item změna na straně dlužníka
    \item nastane na základě dohody dlužníka s třetí osobou, která přijímá jeho dluh a nastupuje
    v závazku na jeho místo (písemně i ústně)
    \item je nutný souhlas věřitele
    \item obsah závazku se nemění
\end{itemize}

\subsubsection{Přistoupení k dluhu}
\begin{itemize}
    \item dohoda věřitele s třetí osobou, na jejímž základě se třetí osoba zavazuje, že se stane
    dlužníkem vedle dlužníka původního (písemně i ústně)
    \item oba dlužníci jsou pak zavázáni solidárně
    \item přistoupení lze realizovat i bez souhlasu původního dlužníka
\end{itemize}

\subsubsection{Převzetí majetku}
\begin{itemize}
    \item převezme-li někdo od zcizitele veškerý majetek nebo jeho poměrnou část, stává se 
    nabyvatel do výše hodnoty nabytého majetku solidárním dlužníkem z dluhů, které s převzatým
    majetkem souvisí
\end{itemize}

\subsubsection{Postoupení smlouvy}
\begin{itemize}
    \item změna v subjektech závazku, kdy jedna za smluvních stran převede svá práva a povinnosti ze
    smlouvy na třetí osobu
    \item postoupená strana (ta strana, která zůstává v závazku) musí s postoupením souhlasit
    \item postoupení je vůči ní účinné od udělení souhlasu
\end{itemize}

\subsection{Změna v obsahu}
\begin{itemize}
    \item ke změně v obsahu závazku může dojít \textbf{na základě ujednání stran}
    \item návrh na změnu musí být přijat \textbf{včas a řádně}
    \item ke změně obsahu závazku může dojít \textbf{novací} nebo \textbf{narovnáním} (ústně nebo 
    písemné)
    \item \textbf{novace} - dohodou stran lze původní závazek zrušit a nahradit jej novým, a nebo může
    původní závazek existovat současně s novým
    \item \textbf{narovnání} - narovnáním lze odstranit spory nebo pochybnosti. Dohodou o narovnání se
    dosavadní závazek ruší a nahradí se novým
\end{itemize}

\newpage

\section{Zajištění dluhu věcně právními zajišťovacími prostředky}

Podstata zajištění spočívá \textbf{v možnosti věřitele uspokojit svoji pohledávku jiným způsobem
v případě, že dlužník svůj dluh nesplní}. O zajišťovací funkci může být řeč jen tehdy,
\textbf{pokud je prostředek způsobilý plnění dluhu nahradit, tj. věřiteli dává jistotu,
že nějaké plnění skutečně obdrží}. Mezi věcně právní zajišťovací prostředky patří \textbf{zástavní,
podzástavní a zadržovací právo}. Věcně-právní zajišťovací prostředky se \textbf{váží na určitou věc}.

\subsection{Význam institutů}
\begin{itemize}
    \item zlepšují postavení (budoucího) věřitele
    \item působí preventivně - odrazují dlužníka od případného porušení povinnosti
    \item v případě porušení povinnosti dlužníkem - uspokojení věřitele z náhradního zdroje 
    (uhrazovací funkce)
    \item utvrzují - zvyšují či mění kvalitu pohledávky
\end{itemize}

\subsection{Princip akcesority}
Vznik a trvání zajišťovacích a utvrzovacích institutů je spjato s existencí a platností daného 
závazku. Určuje, že co se stane se závazkem, stane se i se zajištěním. Když je závazek splněn a 
zanikne, tím zaniká i zajištění. pokud zanikne z jiného důvodu, než že byl splněn, zaniká i zajištění.

\subsection{Princip subsidiarity}
Jejich použití přichází v úvahu podpůrně až tehdy, nedošlo-li k řádnému uspokojení věřitele z
hlavního závazku

\subsection{Zástavní právo}
\begin{itemize}
    \item \textbf{zakládá věřiteli oprávnění v případě, kdy dlužník nesplní dluh řádně a včas,
    uspokojit se z výtěžku peněžní zástavby}
    \item \textbf{předmět zástavy} - každá věc, s níž lze obchodovat, věci hmotné i nehmotné 
    (cenné papíry, obchodní podíl, pohledávka, ...). Věřitel musí o věc pečovat a užívat ji jen se
    svolením dlužníka
    \item \textbf{předmět zajištění} - dluh peněžitý i nepeněžitý
    \item \textbf{vznik zástavního práva} - na základě písemné zástavní smlouvy, na základě
    rozhodnutí soudu nebo orgánu veřejné moci, ze zákona
    \item \textbf{zánik ústavního práva} - zánikem zajištěného dluhu - splněním dluhu dlužníkem nebo
    uspokojením věřitele zpeněžením zástavy
\end{itemize}

\subsection{Podzástavní právo}
\begin{itemize}
    \item ke vzniku doje tehdy, když je zastavena pohledávka, která je již zajištěna zástavním právem 
    - \textbf{věřitel pozastaví svou pohledávku}
    \item sekundární, podzástavní věřitel pak může dosáhnout zpeněžení primární zástavy místo 
    primárního zástavního věřitele
\end{itemize}

\subsection{Zadržovací právo}
\begin{itemize}
    \item podstatou je oprávnění držitele zadržet cizí \textbf{movitou} věc, kterou má oprávněně u
    sebe, a to k zajištění své pohledávky vůči osobě, které by jinak byl tuto věc vydat
    \item může být realizováno např. v případě opravy věci, kterou ale objednatel nebude 
    chtít zaplatit
    \item věřitel musí o věc pečovat a užívat ji jen se svolením dlužníka
\end{itemize}

\newpage

\section{Zajištění dluhu obligačními zajišťovacími prostředky}

Podstata zjištění spočívá \textbf{v možnosti věřitele uspokojit svoji pohledávku jiným způsobem,
v případě že dlužník svůj dluh nesplní}. O zajišťovací funkci může být řeč jen tehdy, \textbf{pokud
je prostředek způsobilý splnění dluhu nahradit}. Mezi obligační zajišťovací prostředky patří 
\textbf{ručení, finanční záruka, zajišťovací převod práva a dohoda o srážkách ze mzdy jiných příjmů}

\subsection{Ručení}
\begin{itemize}
    \item ručení vzniká \textbf{prohlášením třetí osoby vůči věřiteli}, že ho uspokojí, jestliže
    dlužník vůči němu nesplní svůj dluh
    \item musí být učiněno v písemné formě a musí z něj být zřejmé, kdo ručení poskytuje, kdo je 
    dlužníkem, za kterého je ručení poskytnuto a samotný dluh, na který se ručení vztahuje
\end{itemize}

\subsection{Finanční záruka}
\begin{itemize}
    \item \textbf{prohlášení výstavce v písemné záruční listině}, že uspokojí věřitele do výše
    určité částky v případě, že dlužník nesplní svůj závazek vůči věřiteli, a nebo splní-li se 
    jiné, v záruční listině určené, podmínky
    \item záruční listina musí obsahovat identifikaci vystavovatele, věřitele oprávněného na 
    záruku, údaj o zajišťovaném dluhu, maximální výši plnění a podmínky, za nichž dojde k výplatě 
    a doba platnosti záruky
    \item může být peněžitá i nepeněžitá pohledávka
    \item je-li výstavcem banka, jedná se o bankovní záruku
\end{itemize}

\subsection{Zajišťovací převod práva}
\begin{itemize}
    \item písemná smlouva o převodu práva (např. vlastnického) z dlužníka na věřitele
    \item časově omezený převod
\end{itemize}

\subsection{Dohoda o srážkách ze mzdy nebo jiných příjmů}
\begin{itemize}
    \item písemná dohoda mezi dlužníkem a věřitelem, že pohledávka bude uspokojena srážkami ze
    mzdy (či jiných příjmů) dlužníka, a to maximálně ve výši nepřesahující jejich polovinu
\end{itemize}

\subsection{Utvrzení dluhu}
\begin{itemize}
    \item \textbf{utvrzení dluhu} představuje preventivní prostředky zajištění dluhu a současně
    prostředky, které věřiteli ulehčují vymáhání dluhu - NOZ k nim řadí \textbf{smluvní pokutu a
    uznání dluhu}
\end{itemize}

\subsubsection{Smluvní pokuta}
\begin{itemize}
    \item \textbf{strany se dohodnou o právu na její zaplacení v případě porušení smluvní povinnosti}
    \item \textbf{může spočívat i v jiném, než peněžitém plnění, lze sjednat i v jiné než písemné
    formě}
    \item má charakter \textbf{majetkové sankce} - může požadovat pokutu, i kdyby mu porušením
    povinnosti nevznikla škoda a \textbf{paušalizované náhrady škody} - věřitel není oprávněn 
    požadovat kromě smluvní pokuty také náhradu škody
    \item zákon neupravuje výši pokud, nepřiměřeně vysokou \textbf{pokutu může snížit soud}
\end{itemize}

\subsubsection{Uznání dluhu}
\begin{itemize}
    \item dojde-li k uznání dluhu dlužníkem, běží od uznání nová \textbf{promlčecí doba} - 10 let
    \item ,,uzná-li někdo svůj dluh co do důvodu i výše prohlášením učiněným v \textbf{písemné formě},
    má se za to, že dluh v rozsahu uznání a v době uznání trvá"
\end{itemize}

\newpage

\section{Zánik závazků splněním, zánik nesplněného závazku na základě právního jednání}

\begin{itemize}
    \item závazky mohou zaniknout splněním nebo jinými způsoby (není-li závazek splněn)
    \item podle toho, na základě jakých skutečností dochází k zániku, lze rozlišit zánik závazku na
    základě:
    \begin{itemize}
        \item \textbf{jednostranného právního jednání} - např. splnění, odstoupením od smlouvy, 
        zaplacení odstupného
        \item \textbf{dvoustranného právního jednání} - např. dohoda, započtení pohledávek 
        dvoustranným právním jednáním
        \item \textbf{právní událostí} - např. uplynutím doby, smrtí dlužníka nebo věřitele
        \item \textbf{právní skutečnosti} - např. prodlení u fixní smlouvy, neuplatnění práva
        \item \textbf{jiné skutečnosti}
    \end{itemize}
    \item splnění představuje nejčastější způsob zániku závazků
    \item aby mohl závazek takto zaniknout, musí být \textbf{splněn řádně a včas}
\end{itemize}

\subsection{Řádně}
\begin{itemize}
    \item \textbf{místo plnění} - nemusí být zapsáno ve smlouvě, Když smlouva uzavřená mezi stranami 
    neobsahuje určení místa plnění: pro nepeněžitý závazek v místě sídla dlužníka, pro peněžitý
    probíhá plnění v místě sídla věřitele nebo také \textbf{připsáním peněžní částky na účet věřitele}
    \item \textbf{vadné plnění} - není-li závazek splněn řádně, pak nedojde k zániku závazku, ale ke
    změně závazku (úroky)
\end{itemize}

\subsection{Včas}
\begin{itemize}
    \item \textbf{doba plnění} - není nutné ji zahrnout do smlouvy, pokud není sjednána, platí, že
    doba plnění nastává vyzváním věřitele. Ve smlouvě může být sjednáno, že dobu plnění je oprávněn 
    určit dlužník.
    \begin{itemize}
        \item \textbf{doba plnění může být stanovena ve prospěch}:
        \begin{itemize}
            \item \textbf{dlužníka} - dlužník může plnit dříve, než je určen termín plnění a věřitel
            takové plnění nemůže odmítnout jako předčasné. Zároveň věřitel před termínem nemůže plnění
            požadovat (smlouva o dílo)
            \item \textbf{věřitele} - opravňuje věřitele žádat dlužníka o splnění před daným termínem
            a dlužník takovou výzvou věřitele musí plnit závazek, bez výzvy dlužník plnit svůj 
            závazek nesmí
            \item \textbf{obou} - závazek musí být splěn až ve stanovené době plnění, před touto
            dobou nemůže ani věřitel plnění požadovat ani dlužník plnit (kupní smlouva)
        \end{itemize}
    \end{itemize}
    \item \textbf{pozdní plnění} - dlužník se dostává do prodlení, \textbf{pokud svůj dluh neplní 
    řádně a včas}
    \begin{itemize}
        \item věřitel může trvat na splnění
        \item může odstoupit od smlouvy
        \item může požadovat úrok z prodlení a náhradu škody
    \end{itemize}
    \item pokud dlužní vše splní, může požadovat \textbf{důkaz o převzetí} (jména, předmět, místo, čas)
\end{itemize}

\subsection{Částečné plnění}
\begin{itemize}
    \item je možné, \textbf{neodporuje-li to povaze závazku} (smontovaný obráběcí stroj nelze dodat po
    částech, ale 10 strojů lze dodat po kusech)
    \item je-li závazek splněn jen z části, zůstává co do zbytku nesplněn s důsledky, které se váží
    k prodlení s plněním
\end{itemize}

\subsection{Zánik nesplněného závazku na základě právního jednání}
\begin{itemize}
    \item k zániku závazku může dojít i tehdy, nedojde-li k jeho splnění (právní úkon - projev vůle)
\end{itemize}

\subsubsection{Odstoupení od smlouvy}
\begin{itemize}
    \item častým důvodem je porušení smluvních povinností dlužníkem či věřitelem
    \begin{itemize}
        \item \textbf{ze smlouvy} - podmínky, za kterých lze odstoupit od smlouvy, jsou
        stanovené ve smlouvě
        \item \textbf{ze zákona} - odstoupit od uzavřené smlouvy lze ze zákona z důvodu 
        prodlení dlužníka nebo věřitele a z důvodu dodání vadného plnění s neodstranitelnou chybou
    \end{itemize}
    \item \textbf{prodlení dlužníka nebo věřitele}
    \begin{itemize}
        \item \textbf{podstatné porušení smlouvy} - od smlouvy lze odstoupit hned, aniž by bylo třeba
        poskytnout dodatečnou přiměřenou lhůtu k plnění, podmínkou je, že odstoupení je oznámeno 2.
        straně
        \item \textbf{nepodstatné porušení smlouvy} - lze odstoupit za podmínky poskytnutí přiměřené
        lhůty k plnění
        \begin{itemize}
            \item uplynutím dodatečné lhůty pro plnění neznamená samo o sobě zánik závazku a je třeba
            učinit ještě další krok, doručit oznámení o odstoupení 2. straně
            \item jestliže však v poskytnuté přiměřené lhůtě je závazek splněn, účinky odstoupení
            nenastávají
        \end{itemize}
    \end{itemize}
    
    \item účinky odstoupení od smlouvy
    \begin{itemize}
        \item zánik všech práv a povinností vyplývajících ze smlouvy
        \item zůstává nárok na náhradu škody, pokud vznikla škoda porušením smlouvy
        \item lze uplatnit nárok na smluvní pokutu, byla-li sjednána
        \item poskytnuté plnění je třeba vrátit, platí pro obě strany bez ohledu na to, která ze stran
        od smlouvy odstupuje - u peněžitého závazku se plnění vrací spolu s úroky
    \end{itemize}
\end{itemize}

\subsubsection{Odstupné}
\begin{itemize}
    \item musí být sjednáno, zaplacení určité konkrétní částky za účelem zrušení smlouvy
\end{itemize}

\subsubsection{Započtení pohledávek}
\begin{itemize}
    \item dochází k němu tehdy, mají-li věřitel a dlužník vzájemné pohledávky, jejichž plnění je
    stejného druhu a jestliže některý z účastníků učiní vůči druhému projev směřující k započtení.
    \item započtením se obě pohledávky ruší v rozsahu, v jakém se kryjí
    \item započtení pohledávky musí být přípustné, musí se jednat o pohledávku započitatelnou 
    (tj. lze ji uplatnit u soudu), plnění ze vzájemných pohledávek musí být stejného druhu 
    (např. peněžité pohledávky)
\end{itemize}

\subsubsection{Výpověď}
\begin{itemize}
    \item jednostranný právní úkon osoby, který vyjadřuje její vůli nastolit zánik závazku
    \item závazek lze vypovědět, ujednají-li si to strany anebo stanoví-li to zákon
    \item poté závazek zaniká uplynutím výpovědní lhůty (3 měsíce), pokud nějaká je
    \item vypovědět lze smlouvu na dobu neurčitou, smlouvu na dobu neurčitou lze vypovědět, pokud
    se změní okolnosti, za nichž byla smlouva podepsaná
\end{itemize}

\subsubsection{Dohoda}
\begin{itemize}
    \item v případech, kdy žádná ze stran nemá zájem na splnění závazků ze smlouvy
    \item \textbf{zvláštní typy dohody}
    \begin{itemize}
        \item \textbf{novace} - dohoda mezi věřitelem a dlužníkem o tom, že dosavadní závazek bude
        nahrazen závazkem novým
        \item \textbf{narovnání} - úprava jednoho nebo všech práv, která jsou mezi účastníky smlouvy
        sporná či pochybná. dosavadní závazek nahrazen závazkem novým, který plyne z narovnání.
    \end{itemize}
\end{itemize}

\subsubsection{Prominutí dluhu}
\begin{itemize}
    \item zaniká závazek jen jedné strany
    \item věřitel promine dluh nebo jeho část dlužníkovi
\end{itemize}

\subsection{Zánik nesplněného závazku na základě právní události}

\subsubsection{Uplynutí doby}
\begin{itemize}
    \item jedná se o tzv. fixní závazky, kdy je ve smlouvě ujednaná přesná doba plnění
    \item závazek zaniká prodlením dlužníka - uplynutím doby
    \item uplynutím doby zanikají veškerá práva a povinnosti vyplývající ze smlouvy - např. u 
    nájemní smlouvy
    \item \textbf{prekluze} - zánik závazku neuplatněním práva (věřitel zůstal nečinným, ačkoliv měl
    právo uplatnit)
\end{itemize}

\subsubsection{Smrt dlužníka nebo věřitele}
\begin{itemize}
    \item \textbf{smrt dlužníka (FO)} - závazky nezanikají, ale přecházejí na dědice do výše hodnoty
    nabytého dědictví, zanikají jen ty závazky, které mely být plněny osobně dlužníkem (podle smlouvy,
    či z povahy věci)
    \item \textbf{smrt věřitele (FO)} - nezanikají pohledávky a v plném rozsahu přechází na dědice,
    zanikají práva, která byla vázána jen na jeho osobu.
\end{itemize}

\subsubsection{Zánikem právnické osoby}
\begin{itemize}
    \item \textbf{zánik s likvidací} - zanikají všechna práva i závazky
    \item \textbf{zánik bez likvidace} - všechna práva a závazky přecházejí na právního zástupce
\end{itemize}

\subsubsection{Následná nemožnost plnění}
\begin{itemize}
    \item povinnost dlužníka plnit zanikne, stane-li se plnění, které původně při uzavírání smlouvy
    byl možným, následně plněním nemožným 
    \item Pokud by bylo plnění nemožným v době uzavření smlouvy - jedná se o neplatnou smlouvu
    \item \textbf{objektivní} - závazek zaniká - plnit nemůže dlužník, ani nikdo místo něj 
    (nezávislá na osobě a vůli dlužníka)
    \item \textbf{subjektivní} - závazek nezaniká, za dlužníka může plnění poskytnout jiná osoba
    \item povinnost včasné notifikace věřitele. \textbf{důkazní břemeno nese dlužník}
    \item strana, u níž nemožnost plnění nenastala, může nárokovat náhradu vzniklé škody, smluvní
    pokutu a je třeba vrátit již poskytnuté plnění včetně úroků, stejně jako u odstoupení
\end{itemize}

\subsubsection{Splynutí}
\begin{itemize}
    \item právo a povinnost zaniknou, jestliže dluh a pohledávka splyne jedné osobě
    \begin{itemize}
        \item při dědění - věřitel zdědí pohledávku vůči své osobě - bude sám sobě dlužníkem
        \item při sloučení nebo splynutí právnických osob
        \item při prodeji závodu
    \end{itemize}
\end{itemize}

\newpage

\section{Závazky z deliktů (odpovědnost za škodu), bezdůvodné obohacení}

Mentální breakdown právě teď, možná někdy dopíšu, možná ne...

\end{document}

